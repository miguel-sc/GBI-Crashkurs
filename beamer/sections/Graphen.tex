\section{Graphen}

\begin{frame}{Gerichtete Graphen}
  Ein gerichteter Graph G ist definiert als:
  \begin{itemize}
    \item $G=(V,E)$
    \item Knotenmenge V
    \item Kantenmenge E
    \item $E\subseteq V\times V$
  \end{itemize}
\end{frame}


\begin{frame}{Beispielgraphen}
  $V= \lbrace 0,1,2,3,4,5\rbrace $\\ $E = \lbrace (0,0),(0,3),(2,1),(2,5),(5,2)\rbrace$.
  \begin{center}
    \begin{tikzpicture}[->,>=stealth]
      \matrix [matrix of math nodes,ampersand replacement=\&,nodes={circle,draw,minimum size=10mm,inner sep=2pt},row sep=15mm,column sep=15mm] 
      {
        |(0)| 0  \&  |(1)| 1  \&  |(2)| 2  \\
        |(3)| 3  \&  |(4)| 4  \&  |(5)| 5  \\
      };
      \draw (5)to [bend left] (2);
      \draw (2) to [bend left] (5);
      \draw (2) -- (1);
      \path (0) edge [loop left] ();
      \draw (0) -- (3);
    \end{tikzpicture}
  \end{center}
\end{frame}

\begin{frame}{Ungerichtete Graphen}
  Ein ungerichteter Graph ist definiert als:
  \begin{itemize}
    \item $U=(V,E)$
    \item Knotenmenge V
    \item Kantenmenge E
    \item $E\subseteq \{\{x,y\} | x\in V \land y \in V\}$
  \end{itemize}
\end{frame}

\begin{frame}{Beispielgraphen}
  $U=(\{0,1,2,3,4\},\{\{0\},\{1,2\},\{1,3\},\{3,4\}\})$\\
  \begin{figure}
    \centering
    \begin{tikzpicture}[every loop/.style={}]
      \matrix [matrix of math nodes,ampersand replacement=\&,nodes={circle,draw,minimum size=10mm,inner sep=2pt},row sep=15mm,column sep=15mm] 
      {
        |(1)| 1  \&             \&  |(2)| 2  \\
        |(3)| 3  \&  |(0)| 0  \&             \\
                   \&  |(4)| 4  \&             \\
      };
      \draw (1) -- (2);
      \draw (3) -- (4);
      \path (0) edge [loop left] ();
      \draw (1) -- (3);
    \end{tikzpicture}
  \end{figure}
\end{frame}


\begin{frame}{Ungerichtete Graphen}
  Achtung:
  \begin{itemize}
    \item für $x\not=y$ ist $\{x,y\}$ eine zweielementige Menge, ohne eine Festlegung von Reihenfolge
    \item für $x=y$ ist die Menge $\{x,y\}=\{x\}$ eine ein elementige Menge 
  \end{itemize}
\end{frame}

\begin{frame}{Teilgraphen}
  G' ist ein Teilgraph von G, wenn:
  \begin{itemize}
    \item $G=(V,E)$
    \item $G'=(V',E')$
    \item $V'\subseteq V$
    \item $E' \subseteq E \cap V' \times V'$
  \end{itemize}
  Ein Teilgraph besitzt also nur eine Teilmenge der Knoten mit einer Teilmenge an Kanten zwischen diesen Knoten.
\end{frame}

\begin{frame}{Beispielgraphen}
  $G=(\{0,1,2,3,4\},\{(0,0),(0,1),(0,4),(1,3),(2,3),(4,0)\})$\\
  $G'=(\{0,1,2,4\},\{(0,0),(0,1),(4,0)\})$\\
  \begin{figure}
    \centering
    \begin{tikzpicture}[->,>=stealth]
      \matrix [matrix of math nodes,ampersand replacement=\&,nodes={circle,draw,minimum size=8mm,inner sep=2pt},row sep=10mm,column sep=10mm] 
      {
        |(1)| 1  \&             \& |(2)| 2  \\
        |(3)| 3  \&  |(0)| 0  \&            \\
                   \&  |(4)| 4  \&            \\
      };
      \draw (0) -- (1);
      \draw (4)to [bend left] (0);
      \draw (0) to [bend left] (4);
      \draw (2) -- (3);
      \path (0) edge [loop left] ();
      \draw (1) -- (3);
    \end{tikzpicture}
    \qquad
    \begin{tikzpicture}[->,>=stealth]
      \matrix [matrix of math nodes,ampersand replacement=\&,nodes={circle,draw,minimum size=8mm,inner sep=2pt},row sep=10mm,column sep=10mm] 
      {
        |(1)| 1  \&             \&  |(2)| 2  \\
                   \&  |(0)| 0  \&              \\
                   \&  |(4)| 4  \&              \\
      };
      \draw (0) -- (1);
      \draw (4)to [bend left] (0);
      \path (0) edge [loop left] ();
    \end{tikzpicture}
  \end{figure}
\end{frame}

\begin{frame}{Beispielgraphen}
  $G=(\{0,1,2,3,4\},\{(0,0),(0,1),(0,4),(1,3),(2,3),(4,0)\})$\\
  $G'=(\{0,2,3,4\},\{(0,4),(2,3),(4,0)\})$\\
  \begin{figure}
    \centering
    \begin{tikzpicture}[->,>=stealth]
      \matrix [matrix of math nodes,ampersand replacement=\&,nodes={circle,draw,minimum size=8mm,inner sep=2pt},row sep=10mm,column sep=10mm] 
      {
        |(1)| 1  \&             \&  |(2)| 2  \\
        |(3)| 3  \&  |(0)| 0  \&             \\
                   \&  |(4)| 4  \&             \\
      };
      \draw (0) -- (1);
      \draw (4)to [bend left] (0);
      \draw (0) to [bend left] (4);
      \draw (2) -- (3);
      \path (0) edge [loop left] ();
      \draw (1) -- (3);
    \end{tikzpicture}
    \qquad
    \begin{tikzpicture}[->,>=stealth]
      \matrix [matrix of math nodes,ampersand replacement=\&,nodes={circle,draw,minimum size=8mm,inner sep=2pt},row sep=10mm,column sep=10mm] 
      {
                   \&              \& |(2)| 2  \\
        |(3)| 3  \&  |(0)| 0  \&             \\
                   \&  |(4)| 4  \&             \\
      };
      \draw (2) -- (3);
      \draw (4)to [bend left] (0);
      \draw (0) to [bend left] (4);
    \end{tikzpicture}
  \end{figure}
\end{frame}

\begin{frame}{Pfade}
  Für einen Pfad p gilt:
  \begin{itemize}
    \item $p=(v_0,...,v_n) \in V^+$
    \item $(v_i,v_{i+1})\in E$
    \item Bei ungerichteten Graphen nennen wir soetwas einen Weg
  \end{itemize}
\end{frame}


\begin{frame}{Pfade}
  \begin{figure}
    \centering
    \begin{tikzpicture}[->,>=stealth]
      \matrix [matrix of math nodes,ampersand replacement=\&,nodes={circle,draw,minimum size=7mm,inner sep=2pt},row sep=10mm,column sep=10mm] 
      {
        |(1)| 1  \& \& |(2)| 2  \\
        |(3)| 3 \& |(0)| 0 \& \\
        \& |(4)| 4 \& \\
      };
      \draw (0) -- (1);
      \draw (4)to [bend left] (0);
      \draw (0) to [bend left] (4);
      \draw (2) -- (3);
      \path (0) edge [loop left] ();
      \draw (1) -- (3);
    \end{tikzpicture}
  \end{figure}
  \begin{itemize}
    \item $p_1=(2,3)$
    \item $p_2=(0,0,4,0)$
    \item $p_3=(4,0,1,3)$
    \item usw..
  \end{itemize}
\end{frame}


\begin{frame}{Bäume}
  \begin{figure}
    \centering
    \begin{tikzpicture}[->,>=stealth]
      \matrix [matrix of math nodes,ampersand replacement=\&,nodes={circle,draw,minimum size=8mm,inner sep=2pt},row sep=10mm,column sep=10mm] 
      {
        \& |(0)| 0 \&  \\
        |(1)| 1 \&  \& |(2)| 2 \\
        \& |(3)| 3 \& |(4)| 4\\
      };
      \draw (0) -- (1);
      \draw (0) -- (2);
      \draw (2) -- (3);
      \draw (2) -- (4);
    \end{tikzpicture}
  \end{figure}
  \begin{itemize}
    \item eindeutiger Pfad von der Wurzel zu allen Knoten
    \item Bei ungerichteten Bäumen ist die Wurzel nicht eindeutig
  \end{itemize}
\end{frame}

\begin{frame}{Bäume}
  \begin{figure}
    \centering
    \begin{tikzpicture}[->,>=stealth]
      \matrix [matrix of math nodes,ampersand replacement=\&,nodes={circle,draw,minimum size=8mm,inner sep=2pt},row sep=10mm,column sep=10mm] 
      {
        |(0)| 0\& \& \& |(4)| 4\& |(5)|5 \\
        \& |(1)|1 \& |(2)| 2 \& |(6)|6 \& \\
        |(3)| 3\& \& \&\&\\
      };
      \draw (0) -- (1);
      \draw (2) -- (0);
      \draw (2) -- (3);
      \draw (2) -- (4);
      \draw (2) -- (6);
      \draw (4) -- (5);
    \end{tikzpicture}
  \end{figure}
\end{frame}

\begin{frame}{Isomorphe Graphen}
  \begin{itemize}
    \item Isomorphismus: bijektiver Homomorphismus
    \item Zwei Graphen sind isomorph, wenn sich der eine Graph durch Umbennung der Knoten des anderen Graphen bilden lässt.
  \end{itemize}
\end{frame}

\begin{frame}{Isomorphe Graphen}
  \begin{figure}
    \centering
    \begin{tikzpicture}[->,>=stealth]
      \matrix [matrix of math nodes,ampersand replacement=\&,nodes={circle,draw,minimum size=8mm,inner sep=2pt},row sep=10mm,column sep=10mm] 
      {
        \& |(0)| 0 \&  \\
        |(1)| 1 \&  \& |(2)| 2 \\
        \& |(3)| 3 \& |(4)| 4\\
      };
      \draw (0) -- (1);
      \draw (0) -- (2);
      \draw (2) -- (3);
      \draw (2) -- (4);
    \end{tikzpicture}
    \qquad
    \begin{tikzpicture}[->,>=stealth]
      \matrix [matrix of math nodes,ampersand replacement=\&,nodes={circle,draw,minimum size=8mm,inner sep=2pt},row sep=10mm,column sep=10mm] 
      {
        \& |(1)| 1 \&  \\
        |(3)| 3 \&  \& |(2)| 2 \\
        \& |(0)| 0 \& |(4)| 4\\
      };
      \draw (1) -- (3);
      \draw (1) -- (2);
      \draw (2) -- (0);
      \draw (2) -- (4);
    \end{tikzpicture}
  \end{figure}
\end{frame}

\begin{frame}{Isomorphe Graphen}
  \begin{figure}
    \centering
    \begin{tikzpicture}
      \matrix [matrix of math nodes,ampersand replacement=\&,nodes={circle,draw,minimum size=8mm,inner sep=2pt},row sep=10mm,column sep=10mm] 
      {
        |(0)| 0 \& |(1)| 1 \\
        |(2)| 2 \& |(3)| 3 \\
      };
      \draw (0) -- (1);
      \draw (0) -- (2);
      \draw (0) -- (3);
      \draw (1) -- (2);
      \draw (1) -- (3);
      \draw (2) -- (3);
    \end{tikzpicture}
    \qquad
    \begin{tikzpicture}
      \matrix [matrix of math nodes,ampersand replacement=\&,nodes={circle,draw,minimum size=8mm,inner sep=2pt},row sep=10mm,column sep=10mm] 
      {
        \& |(a)| a \&  \\
        \& |(b)| b \&  \\
        |(c)| c \& \& |(d)| d\\
      };
      \draw (a) -- (b);
      \draw (a) -- (c);
      \draw (a) -- (d);
      \draw (b) -- (c);
      \draw (b) -- (d);
      \draw (c) -- (d);
    \end{tikzpicture}
  \end{figure}
\end{frame}

\begin{frame}{Klausuraufgabe: SS 2013 A2}
  Zeichnen Sie alle ungerichteten nicht-isomorphen Graphen mit 5 Knoten, für die gilt:\\
  Genau ein Knoten besitzt Grad 4, all anderen Knoten haben Grad 2.
\end{frame}

\begin{frame}{Klausuraufgabe: SS 2013 A2}
  Genau ein Knoten besitzt Grad 4, all anderen Knoten haben Grad 2.
  \begin{center}
    \begin{tikzpicture}
      \matrix [matrix of math nodes,ampersand replacement=\&,nodes={circle,draw,minimum size=8mm,inner sep=2pt},row sep=10mm,column sep=10mm] 
      {
        \& |(a)| \&  \\
        |(b)| \&  \& |(c)|  \\
        |(d)| \&  \& |(e)|  \\
      };
    \end{tikzpicture}
  \end{center}
\end{frame}

\begin{frame}{Klausuraufgabe: SS 2013 A2}
  Genau ein Knoten besitzt Grad 4, all anderen Knoten haben Grad 2.
  \begin{center}
    \begin{tikzpicture}
      \matrix [matrix of math nodes,ampersand replacement=\&,nodes={circle,draw,minimum size=8mm,inner sep=2pt},row sep=10mm,column sep=10mm] 
      {
        \& |(a)| \&  \\
        |(b)| \&  \& |(c)|  \\
        |(d)| \&  \& |(e)|  \\
      };
      \draw (a) -- (b);
      \draw (a) -- (c);
      \draw (a) -- (d);
      \draw (a) -- (e);
    \end{tikzpicture}
  \end{center}
\end{frame}

\begin{frame}{Klausuraufgabe: SS 2013 A2}
  Genau ein Knoten besitzt Grad 4, all anderen Knoten haben Grad 2.
  \begin{center}
    \begin{tikzpicture}
      \matrix [matrix of math nodes,ampersand replacement=\&,nodes={circle,draw,minimum size=8mm,inner sep=2pt},row sep=10mm,column sep=10mm] 
      {
        \& |(a)| \&  \\
        |(b)| \&  \& |(c)|  \\
        |(d)| \&  \& |(e)|  \\
      };
      \draw (a) -- (b);
      \draw (a) -- (c);
      \draw (a) -- (d);
      \draw (a) -- (e);
      \draw (b) -- (d);
      \draw (c) -- (e);
    \end{tikzpicture}
  \end{center}
\end{frame}

\begin{frame}{Klausuraufgabe: SS 2013 A2}
  Genau ein Knoten besitzt Grad 4, all anderen Knoten haben Grad 2.
  \begin{center}
    \begin{tikzpicture}[every loop/.style={}]
      \matrix [matrix of math nodes,ampersand replacement=\&,nodes={circle,draw,minimum size=8mm,inner sep=2pt},row sep=10mm,column sep=10mm] 
      {
        \& |(a)| \&  \\
        |(b)| \&  \& |(c)|  \\
        |(d)| \&  \& |(e)|  \\
      };
      \path (a) edge [loop left] ();
      \draw (a) -- (b);
      \draw (a) -- (c);
    \end{tikzpicture}
  \end{center}
\end{frame}

\begin{frame}{Klausuraufgabe: SS 2013 A2}
  Genau ein Knoten besitzt Grad 4, all anderen Knoten haben Grad 2.
  \begin{center}
    \begin{tikzpicture}[every loop/.style={}]
      \matrix [matrix of math nodes,ampersand replacement=\&,nodes={circle,draw,minimum size=8mm,inner sep=2pt},row sep=10mm,column sep=10mm] 
      {
        \& |(a)| \&  \\
        |(b)| \&  \& |(c)|  \\
        |(d)| \&  \& |(e)|  \\
      };
      \path (a) edge [loop left] ();
      \draw (a) -- (b);
      \draw (a) -- (c);
      \draw (b) -- (d);
      \draw (d) -- (e);
      \draw (c) -- (e);
    \end{tikzpicture}
  \end{center}
\end{frame}

\begin{frame}{Klausuraufgabe: SS 2013 A2}
  Genau ein Knoten besitzt Grad 4, all anderen Knoten haben Grad 2.
  \begin{center}
    \begin{tikzpicture}[every loop/.style={}]
      \matrix [matrix of math nodes,ampersand replacement=\&,nodes={circle,draw,minimum size=8mm,inner sep=2pt},row sep=10mm,column sep=10mm] 
      {
        \& |(a)| \&  \\
        |(b)| \&  \& |(c)|  \\
        |(d)| \&  \& |(e)|  \\
      };
      \path (a) edge [loop left] ();
      \draw (a) -- (b);
      \draw (a) -- (c);
      \draw (b) -- (d);
      \draw (d) -- (c);
      \path (e) edge [loop left] ();
    \end{tikzpicture}
  \end{center}
\end{frame}

\begin{frame}{Klausuraufgabe: SS 2013 A2}
  Genau ein Knoten besitzt Grad 4, all anderen Knoten haben Grad 2.
  \begin{center}
    \begin{tikzpicture}[every loop/.style={}]
      \matrix [matrix of math nodes,ampersand replacement=\&,nodes={circle,draw,minimum size=8mm,inner sep=2pt},row sep=10mm,column sep=10mm] 
      {
        \& |(a)| \&  \\
        |(b)| \&  \& |(c)|  \\
        |(d)| \&  \& |(e)|  \\
      };
      \path (a) edge [loop left] ();
      \draw (a) -- (b);
      \draw (a) -- (c);
      \draw (b) -- (c);
      \path (e) edge [loop left] ();
      \path (d) edge [loop left] ();
    \end{tikzpicture}
  \end{center}
\end{frame}

\begin{frame}{Relationen}
  \begin{itemize}
    \item $E\subseteq V\times V$
    \item E ist eine Relation.
    \item Wie sieht ein Graph aus mit $E^2,E^3$?
  \end{itemize}
\end{frame}

\begin{frame}{Relationen}
  Links: $G=(V,E)$, Rechts: $G=(V,E^2)$\\
  \begin{figure}
    \centering
    \begin{tikzpicture}[->,>=stealth]
      \matrix [matrix of math nodes,ampersand replacement=\&,nodes={circle,draw,minimum size=8mm,inner sep=2pt},row sep=10mm,column sep=10mm] 
      {
        |(0)| 0 \& |(1)| 1 \\
        |(2)| 2 \& |(3)| 3 \\
      };
      \draw (0) -- (1);
      \draw (3)to [bend left] (0);
      \draw (0) to [bend left] (3);
      \draw (1) -- (2);
    \end{tikzpicture}
    \qquad
    \begin{tikzpicture}[->,>=stealth]
      \matrix [matrix of math nodes,ampersand replacement=\&,nodes={circle,draw,minimum size=8mm,inner sep=2pt},row sep=10mm,column sep=10mm] 
      {
        |(0)| 0 \& |(1)| 1 \\
        |(2)| 2 \& |(3)| 3 \\
      };
      \draw (0) -- (2);
      \path (0) edge [loop left] ();
      \path (3) edge [loop left] ();
      \draw (3) -- (1);
    \end{tikzpicture}
  \end{figure}
  $E^2$ sind also alle Pfade der Länge 2.
\end{frame}

\begin{frame}{Relationen}
  Links: $G=(V,E)$, Rechts: $G=(V,E^*)$\\
  \begin{figure}
    \centering
    \begin{tikzpicture}[->,>=stealth]
      \matrix [matrix of math nodes,ampersand replacement=\&,nodes={circle,draw,minimum size=8mm,inner sep=2pt},row sep=10mm,column sep=10mm] 
      {
        |(0)| 0 \& |(1)| 1 \\
        |(2)| 2 \& |(3)| 3 \\
      };
      \draw (0) -- (1);
      \draw (3)to [bend left] (0);
      \draw (0) to [bend left] (3);
      \draw (1) -- (2);
    \end{tikzpicture}
    \qquad
    \begin{tikzpicture}[->,>=stealth]
      \matrix [matrix of math nodes,ampersand replacement=\&,nodes={circle,draw,minimum size=8mm,inner sep=2pt},row sep=10mm,column sep=10mm] 
      {
        |(0)| 0 \& |(1)| 1 \\
        |(2)| 2 \& |(3)| 3 \\
      };
      \draw (0) -- (2);
      \path (0) edge [loop left] ();
      \path (1) edge [loop left] ();
      \path (2) edge [loop left] ();
      \path (3) edge [loop left] ();
      \draw (3) -- (1);
      \draw (0) -- (1);
      \draw (3)to [bend left] (0);
      \draw (0) to [bend left] (3);
      \draw (3) -- (2);
      \draw (0) -- (2);
      \draw (1) -- (2);
    \end{tikzpicture}
  \end{figure}
\end{frame}

\begin{frame}{Adjazenzmatrix}
  \begin{itemize}
    \item $n\times n$-Matrix bei einem Graphen mit n Knoten
    \item $A_{ij} = \begin{cases}
      1 & \text{ falls } (i,j)\in E \\
      0 & \text{ falls } (i,j)\notin E
    \end{cases}$
    \item $A_{ij}$ gibt also an, ob eine Kante von i nach j existiert.
  \end{itemize}
\end{frame}

\begin{frame}{Adjazenzmatrix}
  \begin{center}
    \begin{tikzpicture}[->,>=stealth]
      \matrix [matrix of math nodes,ampersand replacement=\&,nodes={circle,draw,minimum size=8mm,inner sep=2pt},row sep=15mm,column sep=15mm] 
      {
        |(0)| 0  \&|(1)| 1 \& |(2)| 2  \\
        |(3)| 3 \&  \& |(4)| 4\\
      };
      \draw (4)to [bend left] (2);
      \draw (2) to [bend left] (4);
      \draw (2) -- (1);
      \path (0) edge [loop left] ();
      \draw (0) -- (3);
    \end{tikzpicture}
  \end{center}
  \begin{center}
    $A=\begin{pmatrix}
      1&0&0&1&0\\
      0&0&0&0&0\\
      0&1&0&0&1\\
      0&0&0&0&0\\
      0&0&1&0&0
    \end{pmatrix}$
  \end{center}
\end{frame}

\begin{frame}{Adjazenzmatrix}
  \begin{itemize}
    \item Woran erkennt man Schlingen?
    \item Wie sehen Adjazenzmatrizen bei ungerichteten Graphen aus?
    \item Wie sieht der Graph zu folgenden Adjazenzmatrizen aus?\\
      $A_{1}=\begin{pmatrix}
        1&0&0&0&1\\
        0&0&0&1&0\\
        1&1&0&1&0\\
        1&0&0&0&0\\
        1&0&0&0&1
      \end{pmatrix}$
      $A_{2}=\begin{pmatrix}
        1&0&0&1&1\\
        0&1&1&0&0\\
        0&1&1&1&0\\
        1&0&1&0&0\\
        1&0&0&0&1
      \end{pmatrix}$
  \end{itemize}
\end{frame}

\begin{frame}{Wegematrix}
  \begin{itemize}
    \item $W_{ij} = \begin{cases}
      1 & \text{ falls ein Weg von i nach j existiert} \\
      0 &  \text{sonst}
    \end{cases}$
    \item Beispiel: \\
    \begin{tikzpicture}[->,>=stealth]
      \matrix [matrix of math nodes,ampersand replacement=\&,nodes={circle,draw,minimum size=7mm,inner sep=2pt},row sep=10mm,column sep=10mm] 
      {
        |(1)| 1  \& \& |(2)| 2  \\
        |(3)| 3 \& |(0)| 0 \& \\
        \& |(4)| 4 \& \\
      };
      \draw (0) -- (1);
      \draw (4)to [bend left] (0);
      \draw (0) to [bend left] (4);
      \draw (2) -- (3);
      \path (0) edge [loop left] ();
      \draw (1) -- (3);
    \end{tikzpicture}
    $W=\begin{pmatrix}
      1&1&0&1&1\\
      0&1&0&1&0\\
      0&0&1&1&0\\
      0&0&0&1&0\\
      1&1&0&1&1
    \end{pmatrix}$
  \end{itemize}
\end{frame}

\begin{frame}{Wegematrix}
  Gegeben sei eine $3\times 3$-Matrix, die überall Einsen hat, außer an einer Stelle, die nicht auf der Hauptdiagonalen liegt. Zeigen Sie, dass A nicht die Wegematrix eines Graphen sein kann.\\
  \pause
  Beispiel:
  $W=\begin{pmatrix}
    1&0&1\\
    1&1&1\\
    1&1&1
  \end{pmatrix}$
  \begin{itemize}
    \item $W_{01}=0$, es existiert kein Weg von 0 zu 1
    \item $W_{02}=1$, es existiert ein Weg von 0 zu 2
    \item $W_{21}=1$, es existiert ein Weg von 2 zu 1
  \end{itemize}
\end{frame}

\begin{frame}{Klausuraufgabe: WS 2012/13 A6}
  Für $n\in\mathbb{N}_+$ sei folgender Graph $G_n=(V_n,E_n)$ definiert:\\
  $V_n=\{x|x\subseteq\mathbb{G}_n\land |x|=2\}$,\\
  $E_n=\{\{u,v\}|u\in V,v\in V,u\cap v=\emptyset\}$.
  \begin{itemize}
    \item Zeichnen Sie $G_4$.
    \item Wie viele Kanten hat $G_5$?
    \item Geben Sie die Wegematrix zu $G_3$ an.
  \end{itemize}
\end{frame}

\begin{frame}{Klausuraufgabe: WS 2012/13 A6}
  $V_n=\{x|x\subseteq\mathbb{G}_n\land |x|=2\}$,\\
  $E_n=\{\{u,v\}|u\in V,v\in V,u\cap v=\emptyset\}$.\\
  $V_4,E_4$ gesucht.
  \pause
  \begin{center}
    \begin{tikzpicture}[every loop/.style={}]
      \matrix [matrix of math nodes,ampersand replacement=\&,nodes={circle,draw,minimum size=10mm,inner sep=2pt},row sep=5mm,column sep=25mm] 
      {
        |(0)| \{0,1\}  \& |(3)| \{2,3\} \\
        |(1)| \{0,2\} \& |(4)| \{1,3\} \\
        |(2)| \{0,3\} \&|(5)| \{1,2\} \\
      };
    \end{tikzpicture}
  \end{center}
\end{frame}

\begin{frame}{Klausuraufgabe: WS 2012/13 A6}
  $V_n=\{x|x\subseteq\mathbb{G}_n\land |x|=2\}$,\\
  $E_n=\{\{u,v\}|u\in V,v\in V,u\cap v=\emptyset\}$.\\
  $V_4,E_4$ gesucht.
  \begin{center}
    \begin{tikzpicture}[every loop/.style={}]
      \matrix [matrix of math nodes,ampersand replacement=\&,nodes={circle,draw,minimum size=10mm,inner sep=2pt},row sep=5mm,column sep=25mm] 
      {
        |(0)| \{0,1\}  \& |(3)| \{2,3\} \\
        |(1)| \{0,2\} \& |(4)| \{1,3\} \\
        |(2)| \{0,3\} \&|(5)| \{1,2\} \\
      };
      \draw (0) -- (3);
      \draw (1) -- (4);
      \draw (2) -- (5);
    \end{tikzpicture}
  \end{center}
\end{frame}

\begin{frame}{Klausuraufgabe: WS 2012/13 A6}
  $V_n=\{x|x\subseteq\mathbb{G}_n\land |x|=2\}$,\\
  $E_n=\{\{u,v\}|u\in V,v\in V,u\cap v=\emptyset\}$.\\
  $V_5,E_5$ gesucht.
  \pause
  \begin{center}
    \begin{tikzpicture}[every loop/.style={}]
      \matrix [matrix of math nodes,ampersand replacement=\&,nodes={circle,draw,minimum size=10mm,inner sep=2pt},row sep=15mm,column sep=15mm] 
      {
        |(0)| \{0,1\}  \& |(3)| \{2,3\} \\
        |(1)| \{2,4\} \& |(4)| \{3,4\} \\
      };
      \draw (0) -- (3);
      \draw (0) -- (4);
      \draw (0) -- (1);
    \end{tikzpicture}
  \end{center}
\end{frame}

\begin{frame}{Klausuraufgabe: WS 2012/13 A6}
  \begin{itemize}
    \item Es gibt $5\cdot 4/2$ Knoten
    \item Jeder Knoten hat 3 Kanten
    \item $5\cdot 4/2 \cdot 3=30$
    \item $30/2=15$ Kanten (Doppelzählung)
  \end{itemize}
\end{frame}
