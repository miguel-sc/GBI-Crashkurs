\section{Hoare-Kalkül}

\begin{frame}{Hoare-Tripel}
  Ein Hoare-Tripel $\{P\}S\{Q\}$ besteht aus einer Vorbedingung $P$, einem Programmstück $S$ und einer Nachbedingung $Q$.\\
  Beispiel: 
  \begin{align*}
    &\{x=a\}\\
    &y\leftarrow x\\
    &\{y=a\}
  \end{align*}
\end{frame}

\begin{frame}{Hoare-Tripel}
  \begin{itemize}
    \item HT-A:\\
      Das Hoare-Tripel $\{\sigma_{x/E}(Q)\}x\leftarrow E\{Q\}$ ist gültig.
    \item Beispiel:\\
      \begin{align*}
        &\{?\}\\
        &x\leftarrow E\\
        &\{y=x\}
      \end{align*}
  \end{itemize}
\end{frame}

\begin{frame}{Hoare-Tripel}
  \begin{itemize}
    \item HT-A:\\
      Das Hoare-Tripel $\{\sigma_{x/E}(Q)\}x\leftarrow E\{Q\}$ ist gültig.
    \item Beispiel:\\
      \begin{align*}
        &\{y=E\}\\
        &x\leftarrow E\\
        &\{y=x\}
    \end{align*}
  \end{itemize}
\end{frame}

\begin{frame}{Hoare-Tripel}
  \begin{itemize}
  \item HT-S:\\
    Sind die Hoare-Tripel $\{P\}S_1\{Q\}$ und
    $\{Q\}S_2\{R\}$ gültig, dann auch das Tripel $\{P\}S_1;S_2\{R\}$.
    \item Beispiel:\\
      \begin{align*}
        &\{a\le x\}\\
        &y\leftarrow x\\
        &\{a\le y\}\\
        &\{a\le y\}\\
        &z\leftarrow y\\
        &\{a\le z\}
      \end{align*}
  \end{itemize}
\end{frame}

\begin{frame}{Hoare-Tripel}
  \begin{itemize}
    \item HT-S:\\
      Sind die Hoare-Tripel $\{P\}S_1\{Q\}$ und
      $\{Q\}S_2\{R\}$ gültig, dann auch das Tripel $\{P\}S_1;S_2\{R\}$.
    \item Beispiel:\\
      \begin{align*}
        &\{a\le x\}\\
        &y\leftarrow x\\
        &z\leftarrow y\\
        &\{a\le z\}
      \end{align*}
  \end{itemize}
\end{frame}

\begin{frame}{Hoare-Tripel}
  \begin{itemize}
    \item HT-E:\\
      Ist ein Hoare-Tripel $\{P\}S\{Q\}$ gültig und sind die
      Aussagen $P' \Rightarrow P$ und $Q\Rightarrow Q'$ wahr, dann ist auch das
      Hoare-Tripel $\{P'\}S\{Q'\}$ gültig.
    \item Beispiel: \\
      \begin{align*}
        &\{a\le x\}\\
        &y\leftarrow x\\
        &\{a\le y\}
      \end{align*}
  \end{itemize}
\end{frame}

\begin{frame}{Hoare-Tripel}
  \begin{itemize}
    \item HT-E:\\
      Ist ein Hoare-Tripel $\{P\}S\{Q\}$ gültig und sind die
      Aussagen $P' \Rightarrow P$ und $Q\Rightarrow Q'$ wahr, dann ist auch das
      Hoare-Tripel $\{P'\}S\{Q'\}$ gültig.
    \item Beispiel: \\
      \begin{align*}
        &\{a =x\}\\
        &\{a\le x\}\\
        &y\leftarrow x\\
        &\{a\le y\}\\
        &\{a\le y+1\}
      \end{align*}
  \end{itemize}
\end{frame}

\begin{frame}{Hoare-Tripel}
  \begin{itemize}
    \item HT-E:\\
      Ist ein Hoare-Tripel $\{P\}S\{Q\}$ gültig und sind die
      Aussagen $P' \Rightarrow P$ und $Q\Rightarrow Q'$ wahr, dann ist auch das
      Hoare-Tripel $\{P'\}S\{Q'\}$ gültig.
    \item Beispiel: \\
      \begin{align*}
        &\{a =x\}\\
        &y\leftarrow x\\
        &\{a\le y+1\}
      \end{align*}
  \end{itemize}
\end{frame}

\begin{frame}{Hoare-Tripel}
  HT-I:\\
  Wenn die Hoare-Tripel $\{P \land B\}S_1\{Q\}$ und $\{P \land \lnot B\}S_2\{Q\}$ gültig sind, so ist auch das Hoare-Tripel $\{P\} \textbf{if } B \textbf{ then }S_1\textbf{ else }S_2\textbf{ fi }\{Q\}$ gültig.
\end{frame}

\begin{frame}
  \begin{align*}
    &\{P\}\\
    &\textbf{if } B\\
    &\textbf{then}\\
    & \qquad \{P\land B\}\\
    & \qquad S_1\\
    & \qquad \{Q\}\\
    &\textbf{else}\\
    & \qquad \{P\land \lnot B\}\\
    & \qquad S_2\\
    & \qquad \{Q\}\\
    &\textbf{fi}\\
    &\{Q\}
  \end{align*}
\end{frame}

\begin{frame}
  \begin{align*}
    &\{|x|=15\}\\
    &\textbf{if } x\geq 0\\
    &\textbf{then}\\
    & \qquad \{|x|=15 \land x\geq 0\}\\
    & \qquad x \leftarrow x\\
    & \qquad \{x=15\}\\
    &\textbf{else}\\
    & \qquad \{|x|=15 \land \lnot (x\geq 0)\}\\
    & \qquad x \leftarrow -x\\
    & \qquad \{x=15\}\\
    &\textbf{fi}\\
    &\{x=15\}
  \end{align*}
\end{frame}

\begin{frame}{Beispiel: Minimum}
  \begin{align*}
    &\{x=a \land y=b\}\\
    &\textbf{if } x > y\\
    &\textbf{then}\\
    & \qquad \{...\}\\
    & \qquad z \leftarrow y\\
    & \qquad \{...\}\\
    &\textbf{else}\\
    & \qquad \{...\}\\
    & \qquad z \leftarrow x\\
    & \qquad \{...\}\\
    &\textbf{fi}\\
    &\{z=\text{min}(a,b)\}
  \end{align*}
\end{frame}

\begin{frame}
  \begin{minipage}{\textwidth}
    \begin{align*}
      &\{x=a \land y=b\}\\
      &\textbf{if } x > y\\
      &\textbf{then}\\
      & \qquad \{x=a \land y=b \land x>y\}\\
      & \qquad \{...\}\\
      & \qquad z \leftarrow y\\
      & \qquad \{...\}\\
      &\textbf{else}\\
      & \qquad \{...\}\\
      & \qquad z \leftarrow x\\
      & \qquad \{...\}\\
      &\textbf{fi}\\
      &\{z=\text{min}(a,b)\}
    \end{align*}
  \end{minipage}
\end{frame}

\begin{frame}
  \begin{minipage}{\textwidth}
    \begin{align*}
      &\{x=a \land y=b\}\\
      &\textbf{if } x > y\\
      &\textbf{then}\\
      & \qquad \{x=a \land y=b \land x>y\}\\
      & \qquad \{...\}\\
      & \qquad z \leftarrow y\\
      & \qquad \{...\}\\
      &\textbf{else}\\
      & \qquad \{x=a \land y=b \land \lnot(x>y)\}\\
      & \qquad \{...\}\\
      & \qquad z \leftarrow x\\
      & \qquad \{...\}\\
      &\textbf{fi}\\
      &\{z=\text{min}(a,b)\}
    \end{align*}
  \end{minipage}
\end{frame}

\begin{frame}
  \begin{minipage}{\textwidth}
    \begin{align*}
      &\{x=a \land y=b\}\\
      &\textbf{if } x > y\\
      &\textbf{then}\\
      & \qquad \{x=a \land y=b \land x>y\}\\
      & \qquad \{...\}\\
      & \qquad z \leftarrow y\\
      & \qquad \{z=\text{min}(a,b)\}\\
      &\textbf{else}\\
      & \qquad \{x=a \land y=b \land \lnot(x>y)\}\\
      & \qquad \{...\}\\
      & \qquad z \leftarrow x\\
      & \qquad \{z=\text{min}(a,b)\}\\
      &\textbf{fi}\\
      &\{z=\text{min}(a,b)\}
    \end{align*}
  \end{minipage}
\end{frame}

\begin{frame}
  \begin{minipage}{\textwidth}
    \begin{align*}
      &\{x=a \land y=b\}\\
      &\textbf{if } x > y\\
      &\textbf{then}\\
      & \qquad \{x=a \land y=b \land x>y\}\\
      & \qquad \{y=\text{min}(a,b)\}\\
      & \qquad z \leftarrow y\\
      & \qquad \{z=\text{min}(a,b)\}\\
      &\textbf{else}\\
      & \qquad \{x=a \land y=b \land \lnot(x>y)\}\\
      & \qquad \{x=\text{min}(a,b)\}\\
      & \qquad z \leftarrow x\\
      & \qquad \{z=\text{min}(a,b)\}\\
      &\textbf{fi}\\
      &\{z=\text{min}(a,b)\}
    \end{align*}
  \end{minipage}
\end{frame}

\begin{frame}{Hoare-Tripel}
  HT-W:\\
  Wenn das Hoare-Tripel $\{I \land B\}S\{I\}$ gültig ist, so ist auch das Tripel $\{I\} \textbf{ while } B \textbf{ do }S\textbf{ od }\{I \land \lnot B\}$ gültig.
\end{frame}

\begin{frame}{Beispiel: While-Schleife}
  \begin{align*}
    &\{ I \} \\
    &\textbf{while } B  \\
    &\textbf{do } \\
    &\qquad\{ I\land B \} \\
    &\qquad S \\
    &\qquad\{ I \} \\
    &\textbf{od } \\
    &\{I\land \lnot B\}
  \end{align*}
\end{frame}

\begin{frame}{Beispiel: While-Schleife}
  \begin{align*}
    &\{ x\geq 0 \} \\
    &\textbf{while } x\leq 10  \\
    &\textbf{do } \\
    &\qquad\{ x\geq 0 \land x\leq 10 \} \\
    &\qquad x\leftarrow x+1 \\
    &\qquad\{ x\geq 0 \} \\
    &\textbf{od } \\
    &\{x\geq 0 \land \lnot (x\leq 10)\}
  \end{align*}
\end{frame}

\begin{frame}{Beispiel: While-Schleife}
  \begin{align*}
    &\{x=a \land y=b\}  \\
    &\{ ... \} \\
    &\textbf{while } y\not=0  \\
    &\textbf{do } \\
    &\qquad\{ ... \} \\
    &\qquad y \gets y-1 \\
    &\qquad\{ ... \} \\
    &\qquad x \gets x+1 \\
    &\qquad\{ ... \} \\
    &\textbf{od } \\
    &\{ ... \} \\
    &\{x=a+b\} \\
  \end{align*}
\end{frame}

\begin{frame}
  \begin{align*}
    &\{x=a \land y=b\}  \\
    &\{ x+y=a+b \} \\
    &\textbf{while } y\not=0  \\
    &\textbf{do } \\
    &\qquad\{ ... \} \\
    &\qquad y \gets y-1 \\
    &\qquad\{ ... \} \\
    &\qquad x \gets x+1 \\
    &\qquad\{ ... \} \\
    &\textbf{od } \\
    &\{ ... \} \\
    &\{x=a+b\} \\
  \end{align*}
\end{frame}

\begin{frame}
  \begin{align*}
    &\{x=a \land y=b\}  \\
    &\{ x+y=a+b \} \\
    &\textbf{while } y\not=0  \\
    &\textbf{do } \\
    &\qquad\{ x+y=a+b \land y\not=0 \} \\
    &\qquad\{ ... \} \\
    &\qquad y \gets y-1 \\
    &\qquad\{ ... \} \\
    &\qquad x \gets x+1 \\
    &\qquad\{ ... \} \\
    &\textbf{od } \\
    &\{ ... \} \\
    &\{x=a+b\} \\
  \end{align*}
\end{frame}

\begin{frame}
  \begin{align*}
    &\{x=a \land y=b\}  \\
    &\{ x+y=a+b \} \\
    &\textbf{while } y\not=0  \\
    &\textbf{do } \\
    &\qquad\{ x+y=a+b \land y\not=0 \} \\
    &\qquad\{ ... \} \\
    &\qquad y \gets y-1 \\
    &\qquad\{ ... \} \\
    &\qquad x \gets x+1 \\
    &\qquad\{ x+y=a+b \} \\
    &\textbf{od } \\
    &\{ ... \} \\
    &\{x=a+b\} \\
  \end{align*}
\end{frame}

\begin{frame}
  \begin{align*}
    &\{x=a \land y=b\}  \\
    &\{ x+y=a+b \} \\
    &\textbf{while } y\not=0  \\
    &\textbf{do } \\
    &\qquad\{ x+y=a+b \land y\not=0 \} \\
    &\qquad\{ ... \} \\
    &\qquad y \gets y-1 \\
    &\qquad\{ x+1+y=a+b \} \\
    &\qquad x \gets x+1 \\
    &\qquad\{ x+y=a+b \} \\
    &\textbf{od } \\
    &\{ ... \} \\
    &\{x=a+b\} \\
  \end{align*}
\end{frame}

\begin{frame}
  \begin{align*}
    &\{x=a \land y=b\}  \\
    &\{ x+y=a+b \} \\
    &\textbf{while } y\not=0  \\
    &\textbf{do } \\
    &\qquad\{ x+y=a+b \land y\not=0 \} \\
    &\qquad\{ x+1+y-1=a+b \} \\
    &\qquad y \gets y-1 \\
    &\qquad\{ x+1+y=a+b \} \\
    &\qquad x \gets x+1 \\
    &\qquad\{ x+y=a+b \} \\
    &\textbf{od } \\
    &\{ ... \} \\
    &\{x=a+b\} \\
  \end{align*}
\end{frame}

\begin{frame}
  \begin{align*}
    &\{x=a \land y=b\}  \\
    &\{ x+y=a+b \} \\
    &\textbf{while } y\not=0  \\
    &\textbf{do } \\
    &\qquad\{ x+y=a+b \land y\not=0 \} \\
    &\qquad\{ x+1+y-1=a+b \} \\
    &\qquad y \gets y-1 \\
    &\qquad\{ x+1+y=a+b \} \\
    &\qquad x \gets x+1 \\
    &\qquad\{ x+y=a+b \} \\
    &\textbf{od } \\
    &\{ x+y=a+b \land \lnot(y\not=0) \} \\
    &\{x=a+b\} \\
  \end{align*}
\end{frame}

\begin{frame}{Aufgabenblatt: WS 2015/16 A8.2}
  Beweisen Sie anhand des Hoare-Kalküls, dass das Hoare-Tripel
  \begin{align*}
    &\{x=a\land y=b\}\\
    &\textbf{if } x\ge y\\
    &\textbf{then}\\
    & \qquad z\leftarrow x\\
    & \qquad x\leftarrow y\\
    & \qquad y\leftarrow z\\
    &\textbf{else}\\
    & \qquad x\leftarrow x\\
    &\textbf{fi}\\
    &\{x=min(a,b)\land y=max(a,b)\}
  \end{align*}
  gültig ist.
\end{frame}

\begin{frame}
  \begin{align*}
    &\{x=a\land y=b\}\\
    &\textbf{if } x\ge y\\
    &\textbf{then}\\
    & \qquad z\leftarrow x\\
    & \qquad x\leftarrow y\\
    & \qquad y\leftarrow z\\
    &\{x=min(a,b)\land y=max(a,b)\}\\
    &\textbf{else}\\
    & \qquad x\leftarrow x\\
    &\{x=min(a,b)\land y=max(a,b)\}\\
    &\textbf{fi}\\
    &\{x=min(a,b)\land y=max(a,b)\}
  \end{align*}
\end{frame}

\begin{frame}
  \begin{align*}
    &\{x=a\land y=b\}\\
    &\textbf{if } x\ge y\\
    &\textbf{then}\\
    &\{y=min(a,b)\land x=max(a,b)\}\\
    & \qquad z\leftarrow x\\
    & \qquad x\leftarrow y\\
    & \qquad y\leftarrow z\\
    &\{x=min(a,b)\land y=max(a,b)\}\\
    &\textbf{else}\\
    &\{x=min(a,b)\land y=max(a,b)\}\\
    & \qquad x\leftarrow x\\
    &\{x=min(a,b)\land y=max(a,b)\}\\
    &\textbf{fi}\\
    &\{x=min(a,b)\land y=max(a,b)\}
  \end{align*}
\end{frame}

\begin{frame}
  \begin{align*}
    &\{x=a\land y=b\land x\geq y\}\\
    &\{y=min(a,b)\land x=max(a,b)\}\\
  \end{align*}
  \begin{align*}
    &\{x=a\land y=b\land x< y\}\\
    &\{x=min(a,b)\land y=max(a,b)\}\\
  \end{align*}
\end{frame}
