\section{Prädikatenlogik}

\begin{frame}{Prädikatenlogik}
  Drei Schritte:
  \begin{itemize}
    \item Definiert Terme, die aus Konstanten, Variablen und Funktionssymbolen zusammengesetzt sind.
    \item Mit Hilfe von Relationssymbolen und Termen konstruiert man dann atomare Formeln.
    \item Mit Hilfe von zwei Quantoren werden allgemeine prädikatenlogische Formeln konstruiert.
  \end{itemize}
\end{frame}

\begin{frame}{Beispiel für Terme}
  \begin{itemize}
    \item $c$
    \item $y$
    \item $g(x)$
    \item $f(x,g(z))$
    \item $f(c,g(g(z)))$
  \end{itemize}
\end{frame}

\begin{frame}{Atomare Formeln}
  \begin{itemize}
    \item Relationssymbole mit Alphabet $Rel_{PL}$
    \item kurz R,S,..
    \item $\stackrel{.}{=}$ Relation immer dabei
  \end{itemize}
\end{frame}

\begin{frame}{Atomare Formeln}
  korrekte Formeln:
  \begin{itemize}
    \item $R(y,c,g(x))$
    \item $f(x,y)\stackrel{.}{=}g(z)$
    \item $S(c)$
  \end{itemize}
  syntaktisch falsche Formeln:
  \begin{itemize}
    \item $S(x)\stackrel{.}{=}S(x)$
    \item $f(R(x,c))$
    \item $R(S(x),c,y)$
    \item $x\stackrel{.}{=}y\stackrel{.}{=}z$
  \end{itemize}
\end{frame}

\begin{frame}{Quantoren}
  \begin{itemize}
    \item Allquantor $\forall$
    \item Existenzquantor $\exists$
    \item Klammerregel: Quantoren binden am stärksten
    \item $(\exists x F)$ statt $(\lnot(\forall x (\lnot F ) ) )$
    \item $A_{For}=A_{Rel}\cup \{\lnot,\land,\lor,\rightarrow,\forall,\exists\}$
    \item z.B. $\forall xR(x,y)\land S(x)$
  \end{itemize}
\end{frame}

\begin{frame}{Interpretation}
  Es seien Alphabete $Const_{PL}$, $Fun_{PL}$ und $Rel_{PL}$ gegeben.\\
  Sind eine Interpretation (D,I) und eine Variablenbelegung $\beta$ festgelegt, so kann man
  \begin{itemize}
    \item jedem Term einen Wert aus D und
    \item jeder Formel einen Wahrheitswert zuordnen.
  \end{itemize}
\end{frame}

\begin{frame}{Interpretation von Termen}
  \begin{align*}
    val_{D,I,\beta}(t)= \begin{cases} \beta(x_i), \text{ falls }t=x_i \in Var_{PL} \\ 
    I(c_i), \text{ falls } t=c_i \in Const_{PL} \\
    I(f_i)(val_{D,I,\beta}(t_1),..,val_{D,I,\beta}(t_k)) , \text{ falls } t=f_i(t_1,...,t_k)
    \end{cases}
  \end{align*}
\end{frame}

\begin{frame}{Interpretation von atomaren Formeln}
  \begin{align*}
    val_{D,I,\beta}(R_i(t_1,...,t_k))= \begin{cases} w, \text{ falls }(val_{D,I,\beta}(t_1),...,val_{D,I,\beta}(t_k))\in I(R_i)) \\ 
    f, \text{ falls }(val_{D,I,\beta}(t_1),...,val_{D,I,\beta}(t_k))\notin I(R_i))
    \end{cases}
  \end{align*}
  \begin{align*}
    val_{D,I,\beta}(t_1\stackrel{.}{=}t_2)= \begin{cases} w, \text{ falls }val_{D,I,\beta}(t_1)=val_{D,I,\beta}(t_2) \\ 
    f, \text{ falls }val_{D,I,\beta}(t_1)\neq val_{D,I,\beta}(t_2)
    \end{cases}
  \end{align*}
\end{frame}

\begin{frame}{Quantoren}
  \begin{align*}
    \beta^d_{x_i}:Var_{PL}\rightarrow D: x_j \rightarrow \begin{cases} \beta(x_j), \text{ falls } i\neq j\\ 
    d, \text{ falls } i=j\end{cases}
  \end{align*}
  \begin{align*}
    val_{D,I,\beta}(\forall x_i F)= \begin{cases} w, \text{ falls für jedes $d\in D$ und $\beta '=\beta_{x_i}^d$ :} val_{D,I,\beta '}(F)=w \\ 
    f, \text{ sonst} \end{cases}
  \end{align*}
  \begin{align*}
    val_{D,I,\beta}(\exists x_i F)= \begin{cases} w, \text{ falls für mind. $d\in D$ und $\beta '=\beta_{x_i}^d$ :} val_{D,I,\beta '}(F)=w \\ 
    f, \text{ sonst} \end{cases}
  \end{align*}
\end{frame}

\begin{frame}{Beispiele}
  \begin{itemize}
    \item $D=\mathbb{N}_0$, $I(c)=0$, $I(f)$ Addition, $I(R)$ kleiner oder gleich
    \item $\beta(x)=3$, $\beta(y)=42$
    \item $val_{D,I,\beta}(R(y,c))=?$
  \end{itemize}
\end{frame}

\begin{frame}{Beispiele}
  \begin{itemize}
    \item $D=\mathbb{N}_0$, $I(c)=0$, $I(f)$ Addition, $I(R)$ kleiner oder gleich
    \item $\beta(x)=3$, $\beta(y)=42$
    \item $val_{D,I,\beta}(R(y,c))=f$
    \item weil $\beta(y)> I(c)$
  \end{itemize}
\end{frame}

\begin{frame}{Beispiele}
  \begin{itemize}
    \item $D=\{a,b\}^+$, $I(c)=bb$, $I(f)$ Konkatenation
    \item $I(R)$ hat gleich viele a's
    \item $\beta(x)=a$, $\beta(y)=abba$
    \item $val_{D,I,\beta}(f(f(x,c),y)=?$
    \item $val_{D,I,\beta}(R(f(y,x),c))=?$
  \end{itemize}
\end{frame}

\begin{frame}{Beispiele}
  \begin{itemize}
    \item $D=\{a,b\}^+$, $I(c)=bb$, $I(f)$ Konkatenation
    \item $I(R)$ hat gleich viele a's
    \item $\beta(x)=a$, $\beta(y)=abba$
    \item $val_{D,I,\beta}(f(f(x,c),y)=abbabba$
    \item $val_{D,I,\beta}(R(f(y,x),c))=f$
  \end{itemize}
\end{frame}

\begin{frame}{Allgemeingültigkeit}
  Eine prädikatenlogische Formel heißt allgemeingültig, wenn (D, I) und jede passende Variablenbelegung $\beta$ gilt: $val_{D,I,\beta}(F) = w$.\\
  Bsp.
  \begin{align*}
    (\forall x_i (G\rightarrow H)) \rightarrow((\forall x_i G) \rightarrow(\forall x_i H))
  \end{align*}
\end{frame}

\begin{frame}{freie und gebundene Vorkommen}
  Wenn in einer prädikatenlogischen Formel G in einem Term eine Variable x steht, dann spricht man auch von einem Vorkommen der Variablen x in G. (Die Anwesenheit einer Variablen unmittelbar hinter einem Quantor zählt nicht als Vorkommen.)
\end{frame}

\begin{frame}{freie und gebundene Vorkommen}
  \begin{itemize}
    \item Für jede Formel G, die atomar ist, sind alle Vorkommen von Variablen frei
    \item Für jede Formel G der Form $(\forall x_i H)$ oder $(\exists x_i H)$ ist ist jedes Vorkommen von x in H gebunden
    \item Beispiel: $\forall x(R(x,y)\land \exists y R(x,y))$
  \end{itemize}
\end{frame}

\begin{frame}{Substitutionen}
  Es ist möglich Variablen einer prädikatenlogischen Formel durch Terme zu ersetzen. Eine Substitution ist eine Abbildung $\sigma:Var_{PL}\rightarrow L_{Ter}$ \\
  $\sigma_{\{x/c,y/f(x)\}}$:
  \begin{align*}
    \sigma(x)&=c\\
    \sigma(y)&=f(x)\\
    \sigma(z)&=z , z\notin \{x,y\}
  \end{align*}
\end{frame}

\begin{frame}{Kollisionsfrei}
 Eine Substitution $\sigma$ heiße kollisionsfrei für eine Formel G, wenn für jede Variable $x_i$, die durch $\sigma$ verändert wird (also $\sigma(x_i) \neq x_i$) und jede Stelle eines freien Vorkommens von $x_i$ in G gilt: Diese Stelle liegt nicht im Wirkungsbereich eines Quantors $\forall x_j$ oder $\exists x_j$, wenn $x_j$ eine Variable ist, die in $\sigma(x_i)$ vorkommt.
\end{frame}

\begin{frame}{Beispiel}
  Kollisionsfrei:
  \begin{itemize}
    \item $G=S(x)\land \forall x(R(x,y))$
    \item $\sigma_{\{x/f(y),y/c\}}(G)=S(f(y))\land \forall x(R(x,c))$
  \end{itemize}
  nicht Kollisionsfrei:
  \begin{itemize}
    \item $G=\exists y(R(y,c)\land R(x,c))$
    \item $\sigma_{\{x/f(y),y/c\}}(G)=\exists y(R(y,c)\land R(f(y),c))$
  \end{itemize}
\end{frame}

\begin{frame}{Klausuraufgabe: WS 2015/16 A2}
  Beantworten Sie für jede der folgenden prädikatenlogischen Formeln die Frage: \glqq Ist die Formel allgemeingültig?\grqq
  \begin{itemize}
    \item[(i)] $(\lnot\exists x: P(x)) \leftrightarrow (\forall x:\lnot P(x))$
    \item[(ii)] $(\forall x\exists y\exists z: Q(x,y,z))\rightarrow(\exists y\forall x\exists z:Q(x,y,z))$
    \item[(iii)] $(\exists z\exists y\forall x: Q(x,y,z))\rightarrow (\forall x\exists y\exists z:Q(x,y,z))$
  \end{itemize}
  Dabei ist $P$ ein einstelliges Relationssymbol und $Q$ ein dreistelliges Relationssymbol.
  \pause
  \begin{itemize}
    \item[(i)] Ja
    \pause
    \item[(ii)] Nein
    \pause
    \item[(iii)] Ja
  \end{itemize}
\end{frame}

\begin{frame}{Klausuraufgabe: WS 2015/16 A2}
  Formulieren Sie die folgenden Aussagen als prädikatenlogische Formeln über dem Universum aller Menschen:
  \begin{itemize}
    \item[(i)] Jeder Student außer Tom lächelt.
    \item[(ii)] Jeder mag jeden, der sich nicht selbst mag.
  \end{itemize}
  \pause
  \begin{itemize}
    \item[(i)] $\forall x(student(x)\rightarrow (\lnot Tom(x) \leftrightarrow lächelt(x)))$
    \pause
    \item[(ii)] $\forall x \forall y (\lnot mag(y,y)\rightarrow mag(x,y))$
  \end{itemize}
\end{frame}

\begin{frame}{Klausuraufgabe: WS 2015/16 A2}
  Gegeben sei die prädikatenlogische Formel
  \begin{align*}
    \forall x\forall y (R(x,y)\rightarrow R(f(x),f(y)))
  \end{align*}
  und eine Interpretation $(D,I)$ dafür, wobei das Universum $D$ die Menge $\{a,b\}$ sei und die Interpretationsabbildung $I$ gegeben sei durch $I(f)(a)=b$, $I(f)(b)=a$ und $I(R)=\{(a,a),(a,b)\}$.
  \begin{itemize}
    \item[(i)] Geben Sie den Wahrheitswert der Formel in der Interpretation an.
    \item[(ii)] Erläutern sie kurz Ihre Antwort aus Teil (i):
  \end{itemize}
  \pause
  Wahrheitswert: Falsch.
\end{frame}
