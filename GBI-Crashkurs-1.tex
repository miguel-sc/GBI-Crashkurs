%% LaTeX-Beamer template for KIT design
%% by Erik Burger, Christian Hammer
%% title picture by Klaus Krogmann
%%
%% version 2.1
%%
%% mostly compatible to KIT corporate design v2.0
%% http://intranet.kit.edu/gestaltungsrichtlinien.php
%%
%% Problems, bugs and comments to
%% burger@kit.edu

\documentclass[18pt]{beamer}

%% SLIDE FORMAT

% use 'beamerthemekit' for standard 4:3 ratio
% for widescreen slides (16:9), use 'beamerthemekitwide'

\usepackage{templates/beamerthemekit}
% \usepackage{templates/beamerthemekitwide}

\usepackage[utf8]{inputenc}
\usepackage[T1]{fontenc}
\usepackage{lmodern}
\usepackage[ngerman]{babel}
\usepackage{amsmath}
\usepackage{amssymb}
\usepackage{graphicx} 
\usepackage{booktabs}
\usepackage{mathabx}
\usepackage{eurosym}
\usepackage{nameref}

\makeatletter
\newcommand*{\currentname}{\@currentlabelname}
\makeatother

%% TITLE PICTURE

% if a custom picture is to be used on the title page, copy it into the 'logos'
% directory, in the line below, replace 'mypicture' with the 
% filename (without extension) and uncomment the following line
% (picture proportions: 63 : 20 for standard, 169 : 40 for wide
% *.eps format if you use latex+dvips+ps2pdf, 
% *.jpg/*.png/*.pdf if you use pdflatex)

%\titleimage{mypicture}

%% TITLE LOGO

% for a custom logo on the front page, copy your file into the 'logos'
% directory, insert the filename in the line below and uncomment it

%\titlelogo{mylogo}

% (*.eps format if you use latex+dvips+ps2pdf,
% *.jpg/*.png/*.pdf if you use pdflatex)

%% TikZ INTEGRATION

% use these packages for PCM symbols and UML classes
\usepackage{templates/tikzkit}
\usepackage{templates/tikzuml}
\usetikzlibrary{matrix}
\usepackage{tikz}
\usetikzlibrary{arrows}
\usetikzlibrary{automata}
\usetikzlibrary{shapes.arrows,chains}

% the presentation starts here

\title[GBI Crashkurs]{GBI Crashkurs WS 2016/17}
%\subtitle{Tag 1}
\author{Miguel Santos Correa}
\date{22. Februar 2017}

%\institute{Chair for Software Design and Quality}

% Bibliography

%\usepackage[citestyle=authoryear,bibstyle=numeric,hyperref,backend=biber]{biblatex}
%\addbibresource{templates/example.bib}
%\bibhang1em

\begin{document}

% change the following line to "ngerman" for German style date and logos
\selectlanguage{ngerman}

\setbeamercovered{invisible}

%title page
\begin{frame}
  \begin{center}
    \huge\inserttitle\\
    \vskip 1.75mm
    \normalsize
    \insertauthor\\
    \insertdate\\
    \vskip 1.75mm
    \large
    miguelsantoscorrea@gmail.com\\
    \vskip 1.75mm
    https://github.com/miguel-sc
  \end{center}
\end{frame}

%preface
\begin{frame}{Altklausuren}
  Aufgabentypen:
  \begin{itemize}
    \item Multiple Choice
    \item Vollständige Induktion
    \item Relationen
    \item Kontextfreie Grammatiken
    \item Graphen
    \item Reguläre Ausdrücke / Endliche Akzeptoren
    \item Turingmaschinen
  \end{itemize}
\end{frame}

\begin{frame}{Themen}
  \begin{itemize}
    \item Huffman-Codierung
    \item O-Kalkül
    \item Prädikatenlogik
    \item MIMA
    \item Hoare-Kalkül
  \end{itemize}
\end{frame}

\begin{frame}{Themen}
  \begin{itemize}
    \item viele Themen
    \item sehr oberflächlich
  \end{itemize}
  $\rightarrow$ Skript lernen!
  \pause
  \begin{itemize}
    \item Aufgaben sind nicht schwer
    \item aber nur mit Übung
  \end{itemize}
  $\rightarrow$ Altklausuren rechnen!
\end{frame}

%table of contents
\begin{frame}{Gliederung}
  \tableofcontents
\end{frame}

\AtBeginSection[]
{
  \begin{frame}
    \begin{center}
      \huge \currentname
    \end{center}
  \end{frame}
}

%sections
\section{Relationen}

\begin{frame}{Paare, Tupel}
  Unterschied zwischen Paaren und Mengen:
  \begin{itemize}
    \item $(1,2)\neq(2,1)$
    \item $\{1,2\}=\{2,1\}$
  \end{itemize}
\end{frame}

\begin{frame}{Kartesisches Produkt}
  Menge aller Paare (a,b) mit a $\in$ A und b $\in$ B\\
  $A \times B = \{(a,b) | a \in A\text{ und }b \in B \}$
  \begin{align*}
    \{a,b\}\times\{1,2,3\}=\{(a,1),(a,2),(a,3),(b,1),(b,2),(b,3)\}
  \end{align*}
\end{frame}

\begin{frame}{Relationen}
  Relation R ist eine Teilmenge von $A\times B$ \\
  $R\subseteq A\times B$
  \begin{itemize}
    \item $A\times B=\{(a,1),(a,2),(a,3),(b,1),(b,2),(b,3)\}$
    \item $R_1=\{(a,1),(b,1),(b,2)\}$
    \item $R_2=\{(a,1),(a,3),(b,2),(b,3)\}$
    \item $(b,3)\in R_2$ oder $bR_23$
  \end{itemize}
\end{frame}

\begin{frame}{Eigenschaften von Relationen}
  Linkstotal
  \begin{itemize}
    \item $\forall a \in A \exists b \in B : (a,b)\in R$
    \item Jedes Element aus A steht in Relation zu mindestens einem Element aus B
  \end{itemize}
  \begin{center}
    \begin{tikzpicture}[->,>=stealth]
      \matrix [matrix of math nodes,ampersand replacement=\&,nodes={circle,draw,minimum size=10mm,inner sep=2pt},row sep=5mm,column sep=25mm] 
      {
        |(0)| 0  \&  |(3)| 0  \\
        |(1)| 1  \&  |(4)| 1  \\
        |(2)| 2  \&  |(5)| 2  \\
      };
      \draw (0) -- (4);
      \draw (1) -- (3);
      \draw (1) -- (4);
      \draw (2) -- (3);
    \end{tikzpicture}
  \end{center}
\end{frame}

\begin{frame}{Eigenschaften von Relationen}
  Rechtstotal (surjektiv)
  \begin{itemize}
    \item $\forall b \in B \exists a \in A : (a,b)\in R$
    \item Jedes Element aus B steht in Relation zu mindestens einem Element aus A
  \end{itemize}
  \begin{center}
    \begin{tikzpicture}[->,>=stealth]
      \matrix [matrix of math nodes,ampersand replacement=\&,nodes={circle,draw,minimum size=10mm,inner sep=2pt},row sep=5mm,column sep=25mm] 
      {
        |(0)| 0  \&  |(3)| 0  \\
        |(1)| 1  \&  |(4)| 1  \\
        |(2)| 2  \&  |(5)| 2  \\
      };
      \draw (0) -- (4);
      \draw (1) -- (3);
      \draw (1) -- (5);
    \end{tikzpicture}
  \end{center}
\end{frame}

\begin{frame}{Eigenschaften von Relationen}
  Linkseindeutig (injektiv)
  \begin{itemize}
    \item $\forall (a_1,b_1),(a_2,b_2) \in R: a_1\neq a_2 \Rightarrow b_1 \neq b_2$
    \item Jedes Element aus B steht höchstens mit einem Element aus A in Relation
  \end{itemize}
  \begin{center}
    \begin{tikzpicture}[->,>=stealth]
      \matrix [matrix of math nodes,ampersand replacement=\&,nodes={circle,draw,minimum size=10mm,inner sep=2pt},row sep=5mm,column sep=25mm] 
      {
        |(0)| 0  \&  |(3)| 0  \\
        |(1)| 1  \&  |(4)| 1  \\
        |(2)| 2  \&  |(5)| 2  \\
      };
      \draw (0) -- (3);
      \draw (1) -- (4);
      \draw (1) -- (5);
    \end{tikzpicture}
  \end{center}
\end{frame}

\begin{frame}{Eigenschaften von Relationen}
  Rechtseindeutig
  \begin{itemize}
    \item $\forall (a_1,b_1),(a_2,b_2) \in R: b_1\neq b_2 \Rightarrow a_1 \neq a_2$
    \item Jedes Element aus A steht höchstens mit einem Element aus B in Relation
  \end{itemize}
  \begin{center}
    \begin{tikzpicture}[->,>=stealth]
      \matrix [matrix of math nodes,ampersand replacement=\&,nodes={circle,draw,minimum size=10mm,inner sep=2pt},row sep=5mm,column sep=25mm] 
      {
        |(0)| 0  \&  |(3)| 0  \\
        |(1)| 1  \&  |(4)| 1  \\
        |(2)| 2  \&  |(5)| 2  \\
      };
      \draw (0) -- (3);
      \draw (1) -- (3);
    \end{tikzpicture}
  \end{center}
\end{frame}

\begin{frame}{Funktionen}
  \begin{itemize}
    \item linkstotale und rechtseindeutige Relationen nennt man Funktionen
    \item andere Schreibweise: $f: A \rightarrow B$
    \item Definitionsbereich A, Zielbereich B, Bildbereich f(A)
    \item linkseindeutige Funktionen nennt man injektiv
    \item rechtstotale Funktionen nennt man surjektiv
    \item injektive und surjektive Funktionen nennt man bijektiv
  \end{itemize}
\end{frame}

\begin{frame}{Produkt von Relationen}
  \begin{itemize}
    \item $R\subseteq M_1 \times M_2$ und $S\subseteq M_2 \times M_3$
    \item $S\circ R=\{(x,z)|\exists y \in M_2: (x,y) \in R \land (y,z) \in S\}$
    \item Beispiel:\begin{align*}&\{(1,2),(2,2),(1,1)\}\circ \{(1,1),(2,1),(3,3)\}\\ &=\{(1,2),(1,1),(2,2),(2,1)\}\end{align*}
  \end{itemize}
\end{frame}

\begin{frame}{Produkt von Relationen}
  \begin{align*}
    &\{(1,2),(2,2),(1,1)\}\circ \{(1,1),(2,1),(3,3)\}\\
    &=\{(1,2),(1,1),(2,2),(2,1)\}
  \end{align*}
  \begin{center}
    \begin{tikzpicture}[->,>=stealth]
      \matrix [matrix of math nodes,ampersand replacement=\&,nodes={circle,draw,minimum size=10mm,inner sep=2pt},row sep=5mm,column sep=25mm] 
      {
        |(1)| 1  \&  |(4)| 1  \&  |(7)| 1  \\
        |(2)| 2  \&  |(5)| 2  \&  |(8)| 2  \\
        |(3)| 3  \&  |(6)| 3  \&  |(9)| 3  \\
      };
      \draw (1) -- (4);
      \draw (2) -- (4);
      \draw (3) -- (6);
      \draw (4) -- (7);
      \draw (4) -- (8);
      \draw (5) -- (8);
    \end{tikzpicture}
  \end{center}
\end{frame}

\begin{frame}{Produkt von Relationen}
  \begin{align*}
    &\{(1,2),(2,2),(1,1)\}\circ \{(1,1),(2,1),(3,3)\}\\
    &=\{(1,2),(1,1),(2,2),(2,1)\}
  \end{align*}
  \begin{center}
    \begin{tikzpicture}[->,>=stealth]
      \matrix [matrix of math nodes,ampersand replacement=\&,nodes={circle,draw,minimum size=10mm,inner sep=2pt},row sep=5mm,column sep=25mm] 
      {
        |(1)| 1  \&  \&  |(7)| 1  \\
        |(2)| 2  \&  \&  |(8)| 2  \\
        |(3)| 3  \&  \&  |(9)| 3  \\
      };
      \draw (1) -- (7);
      \draw (1) -- (8);
      \draw (2) -- (7);
      \draw (2) -- (8);
    \end{tikzpicture}
  \end{center}
\end{frame}

\begin{frame}{Klausuraufgabe: SS 2012 A2}
  $\mathbb{G}_n=\{0,1,...,n-1\}$
  \begin{itemize}
    \item Geben Sie (graphisch) eine Relation $R_a\subseteq \mathbb{G}_4\times \mathbb{G}_2$ an, so dass $R_a$ rechtstotal und rechtseindeutig, aber nicht linkstotal und nicht linkseindeutig ist.
    \item Wie viele solcher Relationen $R_a$ gibt es?
    \item Geben Sie (in Mengenschreibweise) eine Relation $R_b\subseteq\mathbb{G}_2\times\mathbb{G}_4$ an, so dass $R_b\circ R_a$ rechtstotal und linkseindeutig ist.
  \end{itemize}
\end{frame}

\begin{frame}{Klausuraufgabe: SS 2012 A2}
  rechtstotal und rechtseindeutig, aber nicht linkstotal und nicht linkseindeutig
  \begin{center}
    \begin{tikzpicture}[->,>=stealth]
      \matrix [matrix of math nodes,ampersand replacement=\&,nodes={circle,draw,minimum size=10mm,inner sep=2pt},row sep=5mm,column sep=25mm] 
      {
        |(0)| 0  \&  |(4)| 0  \\
        |(1)| 1  \&  |(5)| 1  \\
        |(2)| 2  \&  \\
        |(3)| 3  \&  \\
      };
    \end{tikzpicture}
  \end{center}
\end{frame}

\begin{frame}{Klausuraufgabe: SS 2012 A2}
  rechtstotal und rechtseindeutig, aber nicht linkstotal und nicht linkseindeutig
  \begin{center}
    \begin{tikzpicture}[->,>=stealth]
      \matrix [matrix of math nodes,ampersand replacement=\&,nodes={circle,draw,minimum size=10mm,inner sep=2pt},row sep=5mm,column sep=25mm] 
      {
        |(0)| 0  \& |(4)| 0  \\
        |(1)| 1  \& |(5)| 1  \\
        |(2)| 2  \&  \\
        |(3)| 3  \&  \\
      };
      \draw (0) -- (4);
      \draw (1) -- (5);
    \end{tikzpicture}
  \end{center}
\end{frame}

\begin{frame}{Klausuraufgabe: SS 2012 A2}
  rechtstotal und rechtseindeutig, aber nicht linkstotal und nicht linkseindeutig
  \begin{center}
    \begin{tikzpicture}[->,>=stealth]
      \matrix [matrix of math nodes,ampersand replacement=\&,nodes={circle,draw,minimum size=10mm,inner sep=2pt},row sep=5mm,column sep=25mm] 
      {
        |(0)| 0  \&  |(4)| 0  \\
        |(1)| 1  \&  |(5)| 1  \\
        |(2)| 2  \&  \\
        |(3)| 3  \&  \\
      };
      \draw (0) -- (4);
      \draw (1) -- (5);
      \draw (2) -- (5);
    \end{tikzpicture}
  \end{center}
\end{frame}

\begin{frame}{Klausuraufgabe: SS 2012 A2}
  Geben Sie (in Mengenschreibweise) eine Relation $R_b\subseteq\mathbb{G}_2\times\mathbb{G}_4$ an, so dass $R_b\circ R_a$ rechtstotal und linkseindeutig ist.
  \begin{center}
    \begin{tikzpicture}[->,>=stealth]
      \matrix [matrix of math nodes,ampersand replacement=\&,nodes={circle,draw,minimum size=10mm,inner sep=2pt},row sep=5mm,column sep=25mm] 
      {
        |(0)| 0  \& |(4)| 0  \&  |(6)| 0  \\
        |(1)| 1  \& |(5)| 1  \&  |(7)| 1  \\
        |(2)| 2  \&  \&  |(8)| 2  \\
        |(3)| 3  \&  \&  |(9)| 3  \\
      };
      \draw (0) -- (4);
      \draw (1) -- (5);
      \draw (2) -- (5);
    \end{tikzpicture}
  \end{center}
\end{frame}

\begin{frame}{Klausuraufgabe: SS 2012 A2}
  \begin{center}
    \begin{tikzpicture}[->,>=stealth]
      \matrix [matrix of math nodes,ampersand replacement=\&,nodes={circle,draw,minimum size=10mm,inner sep=2pt},row sep=5mm,column sep=25mm] 
      {
        |(0)| 0  \&  |(4)| 0  \&  |(6)| 0  \\
        |(1)| 1  \&  |(5)| 1  \&  |(7)| 1  \\
        |(2)| 2  \&  \&  |(8)| 2  \\
        |(3)| 3  \&  \&  |(9)| 3  \\
      };
      \draw (0) -- (4);
      \draw (4) -- (6);
      \draw (4) -- (7);
      \draw (4) -- (8);
      \draw (4) -- (9);
      \draw (1) -- (5);
      \draw (2) -- (5);
    \end{tikzpicture}
  \end{center}
  $R_b=\{(0,0),(0,1),(0,2),(0,3)\}$
\end{frame}

\begin{frame}{Klausuraufgabe: SS 2012 A2}
  Geben Sie (in Mengenschreibweise) eine Relation $R_b\subseteq\mathbb{G}_2\times\mathbb{G}_4$ an, so dass $R_b\circ R_a$ rechtstotal und linkseindeutig ist.
  \begin{center}
    \begin{tikzpicture}[->,>=stealth]
      \matrix [matrix of math nodes,ampersand replacement=\&,nodes={circle,draw,minimum size=10mm,inner sep=2pt},row sep=5mm,column sep=25mm] 
      {
        |(0)| 0  \&  \&  |(6)| 0  \\
        |(1)| 1  \&  \&  |(7)| 1  \\
        |(2)| 2  \&  \&  |(8)| 2  \\
        |(3)| 3  \&  \&  |(9)| 3  \\
      };
      \draw (0) -- (6);
      \draw (0) -- (7);
      \draw (0) -- (8);
      \draw (0) -- (9);
    \end{tikzpicture}
  \end{center}
\end{frame}

\begin{frame}{Reflexivität}
  \begin{itemize}
    \item $R\subseteq M\times M$
    \item R ist reflexiv, wenn: $\Rightarrow \forall x\in M: (x,x)\in R$
    \item Jedes Element steht in Relation zu sich selbst.
  \end{itemize}
  \begin{center}
    \begin{tikzpicture}[->,>=stealth]
      \matrix [matrix of math nodes,ampersand replacement=\&,nodes={circle,draw,minimum size=10mm,inner sep=2pt},row sep=15mm,column sep=15mm] 
      {
        |(0)| 0  \&  |(2)| 2  \\
        |(1)| 1  \&  |(3)| 3  \\
      };
      \path (0) edge [loop left] ();
      \path (1) edge [loop left] ();
      \path (2) edge [loop left] ();
      \path (3) edge [loop left] ();
    \end{tikzpicture}
  \end{center}
\end{frame}

\begin{frame}{Transitivität}
  \begin{itemize}
     \item $R\subseteq M\times M$
     \item R ist transitiv, wenn: $\forall x,y,z \in M:(x,y)\in R \land (y,z)\in R \Rightarrow (x,z)\in R$
  \end{itemize}
  \begin{center}
    \begin{tikzpicture}[->,>=stealth]
      \matrix [matrix of math nodes,ampersand replacement=\&,nodes={circle,draw,minimum size=10mm,inner sep=2pt},row sep=15mm,column sep=15mm] 
      {
        |(0)| 0  \&  |(2)| 2  \\
        |(1)| 1  \&  |(3)| 3  \\
      };
      \draw (0) -- (1);
      \draw (1) -- (2);
      \draw (0) -- (2);
    \end{tikzpicture}
  \end{center}
\end{frame}

\begin{frame}{Symmetrie}
  \begin{itemize}
    \item $R\subseteq M\times M$
    \item R ist symmetrisch, wenn: $\forall x,y \in M:(x,y)\in R  \Rightarrow (y,x)\in R$
  \end{itemize}
  \begin{center}
    \begin{tikzpicture}[->,>=stealth]
      \matrix [matrix of math nodes,ampersand replacement=\&,nodes={circle,draw,minimum size=10mm,inner sep=2pt},row sep=15mm,column sep=15mm] 
      {
        |(0)| 0  \&  |(2)| 2  \\
        |(1)| 1  \&  |(3)| 3  \\
      };
      \draw (0) -- (1);
      \draw (1) -- (0);
      \draw (0) -- (2);
      \draw(2) -- (0);
    \end{tikzpicture}
  \end{center}
\end{frame}

\begin{frame}{Äquivalenzrelation}
  \begin{itemize}
    \item $R\subseteq M\times M$
    \item R ist reflexiv, symmetrisch und transitiv
  \end{itemize}
  \begin{center}
    \begin{tikzpicture}[->,>=stealth]
      \matrix [matrix of math nodes,ampersand replacement=\&,nodes={circle,draw,minimum size=10mm,inner sep=2pt},row sep=15mm,column sep=15mm] 
      {
        |(0)| 0  \&  |(3)| 3  \& |(1)| 1  \\
        |(6)| 6  \&  |(4)| 4  \& |(7)| 7  \\
        |(2)| 2  \&  |(5)| 5  \&  \\
      };
      \draw (0) -- (3);
      \draw (3) -- (0);
      \draw (0) -- (6);
      \draw(6) -- (0);
      \draw (3) -- (6);
      \draw (6) -- (3);
      \draw (1) -- (4);
      \draw(4) -- (1);
      \draw (2) -- (5);
      \draw (5) -- (2);
      \draw (1) -- (7);
      \draw(7) -- (1);
      \draw (4) -- (7);
      \draw(7) -- (4);
      \path (0) edge [loop left] ();
      \path (1) edge [loop left] ();
      \path (2) edge [loop left] ();
      \path (3) edge [loop below] ();
      \path (4) edge [loop left] ();
      \path (5) edge [loop above] ();
      \path (6) edge [loop left] ();
      \path (7) edge [loop below] ();
    \end{tikzpicture}
  \end{center}
\end{frame}

\section{Kontextfreie Grammatiken}

\begin{frame}{Kontextfreie Grammatiken}
  \begin{itemize}
    \item G=(N,T,S,P)
    \item N: Menge der Nichtterminalsymbole
    \item T: Menge der Terminalsymbole
    \item S: Startsymbol
    \item P: Produktionsmenge
  \end{itemize}
\end{frame}

\begin{frame}{Beispiele}
  \begin{itemize}
    \item $G_1=(\{X,Y\},\{a\},Y,\{X\rightarrow \epsilon,Y\rightarrow aY|X\})$
    \item $G_2=(\{S\},\{a,b\},S,\{S\rightarrow \epsilon |aSa|bSb\})$
    \item $G_3=(\{X\},\{a\},X,P)$\\
      $P=\{X\rightarrow aX\}$
  \end{itemize}
\end{frame}

\begin{frame}{Ableitungen}
  \begin{itemize}
    \item $G_2=(\{S\},\{a,b\},S,\{S\rightarrow \epsilon |aSa|bSb\})$
    \item $S\Rightarrow aSa \Rightarrow aaSaa \Rightarrow aabSbaa \Rightarrow aab \epsilon baa =aabbaa$
    \item $S\Rightarrow bSb \Rightarrow baSab \Rightarrow baab$
    \item $S\Rightarrow \epsilon$
  \end{itemize}
\end{frame}

\begin{frame}{Ableitungsbaum}
  $G=(\{X,Y\},\{a,b\},X,\{X\rightarrow aX|bY,Y\rightarrow Yb|b\})$
  \begin{center}
    \begin{tikzpicture}[yscale=0.8]
      [level 1/.style={sibling distance=20mm},
      level 2/.style={sibling distance=20mm},
      level 3/.style={sibling distance=20mm}]
      \node {X}
      child { 
        node {a}
       } 
      child {
        node {X}
        child {
          node {b}
        }
        child {
          node {Y}
          child {
            node{}edge from parent[draw=none]
          }
          child {
            node {Y}
            child {
              node {b}
            }
          }
          child {
            node {b}
          }
        }
        child {
          node{}edge from parent[draw=none]
        }
      }
      child {
        node{}edge from parent[draw=none]
      };
    \end{tikzpicture}
  \end{center}
  $X\Rightarrow aX\Rightarrow abY \Rightarrow abYb \Rightarrow abbb$
\end{frame}

\begin{frame}
  $G=(\{X,Y\},\{a,b\},X,\{X\rightarrow XX|a|b\})$
  \begin{center}
    \begin{tikzpicture}
      [level 1/.style={sibling distance=20mm},
      level 2/.style={sibling distance=20mm},
      level 3/.style={sibling distance=20mm}]
      \node {X}
      child {
        node {X}
        child {
          node{a}
        }
      } 
      child {
        node{X}
        child {
          node{X}
          child {
            node{X}
            child {
              node{a}
            }
          }
          child {
            node{X}
            child {
              node{a}
            }
          }
        }
        child {
          node{X}
          child {
            node{b}
          }
        }
      };
    \end{tikzpicture}
  \end{center}
  z.B. $X\Rightarrow XX \Rightarrow aX \Rightarrow aXX \Rightarrow aXb \Rightarrow aXXb \Rightarrow aaXb \Rightarrow aaab$
\end{frame}

\begin{frame}{Klausuraufgabe: WS 2014/15 A2}
  Eine Folge $(L_n)_{n\in\mathbb{N}_0}$ formaler Sprachen sei wie folgt definiert:
  \begin{align*}
    L_0&=\{\}\\
    \forall i\in \mathbb{N}_0:L_{i+1}&=\{ba\}L_i\{ab\}\cup\{b\}
  \end{align*}
  \begin{itemize}
    \item Geben Sie $L_1$, $L_2$ und $L_3$ an.
    \item Geben Sie $L=\bigcup\limits_{i=0}^\infty L_i$ an.
    \item Geben Sie eine kontextfreie Grammatik $G$ mit $L(G)=L$ an.
    \item Zeichnen sie passend zu Ihrer Grammatik einen Ableitungsbaum eines Wortes $w\in L_3 \setminus L_2$.
  \end{itemize}
\end{frame}

\begin{frame}{Klausuraufgabe: WS 2014/15 A2}
  \begin{align*}
    L_0&=\{\}\\
    \forall i\in \mathbb{N}_0:L_{i+1}&=\{ba\}L_i\{ab\}\cup\{b\}
  \end{align*}
  \begin{itemize}
    \item $L_1=\{b\}$
    \pause
    \item $L_2=\{babab,b\}$
    \pause
    \item $L_3=\{babababab,babab,b\}$
    \pause
    \item $L=\bigcup\limits_{i=0}^\infty L_i=$\pause $\{(ba)^nb(ab)^n|n\in \mathbb{N}_0\}$
  \end{itemize}
\end{frame}

\begin{frame}{Klausuraufgabe: WS 2014/15 A2}
  \begin{align*}
    L=\{(ba)^nb(ab)^n|n\in \mathbb{N}_0\}
  \end{align*}
  \pause
  \begin{align*}
    G=(\{S\},\{a,b\},S,\{S\rightarrow baSab | b\})
  \end{align*}
\end{frame}

\begin{frame}{Klausuraufgabe: WS 2014/15 A2}
  $G=(\{S\},\{a,b\},S,\{S\rightarrow baSab | b\})$
  \begin{center}
    \begin{tikzpicture}
      [level 1/.style={sibling distance=15mm},
      level 2/.style={sibling distance=15mm},
      level 3/.style={sibling distance=15mm}]
      \node {S}
      child {
        node {b}
      }
      child {
        node {a}
      } 
      child {
        node{S}
        child {
          node {b}
        }
        child {
          node {a}
        }
        child {
          node{S}
          child {
            node {b}
          }
        }
        child {
          node {a}
        }
        child {
          node {b}
        }
      }
      child {
        node{a}
      }
      child {
        node{b}
      };
    \end{tikzpicture}
  \end{center}
  $S\Rightarrow baSab\Rightarrow babaSabab \Rightarrow babababab$
\end{frame}

\begin{frame}{Klausuraufgabe: WS 2015/16 A6}
  Gegeben sei die kontextfreie Grammatik $G=(\{S,A,B\},\{a,b\},S,P)$ mit der Produktionsmenge
  \begin{align*}
    P=\{S&\rightarrow ASB | A | B,\\
    A&\rightarrow Aa | \epsilon,\\
    B&\rightarrow bB | \epsilon\}.
  \end{align*}
  \begin{itemize}
    \item Geben Sie zwei verschiedene Ableitungsbäume der Grammatik für das Wort $aab$ an.
    \item Geben Sie einen regulären Ausdruck an, der die von $G$ erzeugte Sprache $L(G)$ beschreibt.
    \item Geben Sie eine kontextfreie Grammatik $G'$ an, die die Sprache $(L(G))^*$ erzeugt.
  \end{itemize}
\end{frame}

\begin{frame}{Klausuraufgabe: WS 2015/16 A6}
  \begin{align*}
    P=\{S&\rightarrow ASB | A | B,\\
    A&\rightarrow Aa | \epsilon,\\
    B&\rightarrow bB | \epsilon\}.
  \end{align*}
  Ableitungen für das Wort $aab$
  \begin{itemize}
    \pause
    \item $S\Rightarrow ASB \Rightarrow AAB\Rightarrow ...$
    \pause
    \item $S\Rightarrow ASB \Rightarrow AASBB\Rightarrow^2 A\epsilon S\epsilon B\Rightarrow ...$
  \end{itemize}
\end{frame}

\begin{frame}{Klausuraufgabe: WS 2015/16 A6}
  \begin{center}
    \begin{tikzpicture}
      [level 1/.style={sibling distance=15mm},
      level 2/.style={sibling distance=15mm},
      level 3/.style={sibling distance=15mm}]
      \node {S}
      child {
        node {A}
        child {
          node {A}
          child {
            node{$\epsilon$}
          }
        }
        child {
          node{a}
        }
      }
      child {
        node{S}
        child {
          node{A}
          child {
            node {A}
            child {
              node{$\epsilon$}
            }
          }
          child {
            node{a}
          }
        }
      }
      child {
        node{B}
        child {
          node{b}
        }
        child {
          node{B}
          child {
            node{$\epsilon$}
          }
        }
      };
    \end{tikzpicture}
  \end{center}
\end{frame}

\begin{frame}{Klausuraufgabe: WS 2015/16 A6}
  \begin{center}
    \begin{tikzpicture}
      [level 1/.style={sibling distance=25mm},
      level 2/.style={sibling distance=10mm},
      level 3/.style={sibling distance=10mm}]
      \node {S}
      child {
        node {A}
        child {
          node {A}
          child {
            node{$\epsilon$}
          }
        }
        child {
          node{a}
        }
      } 
      child {
        node{S}
        child {
          node{A}
          child {
            node {A}
            child {
              node{$\epsilon$}
            }
          }
          child {
            node{a}
          }
        }
        child {
          node{S}
          child {
            node{A}
            child {
              node{$\epsilon$}
            }
          }
        }
        child {
          node{B}
          child {
            node{$\epsilon$}
          }
        }
      }
      child {
        node{B}
        child {
          node{b}
        }
        child {
          node{B}
          child {
            node{$\epsilon$}
          }
        }
      };
    \end{tikzpicture}
  \end{center}
\end{frame}

\begin{frame}{Klausuraufgabe: WS 2015/16 A6}
  \begin{align*}
    P=\{S&\rightarrow ASB | A | B,\\
    A&\rightarrow Aa | \epsilon,\\
    B&\rightarrow bB | \epsilon\}.
  \end{align*}
  \begin{itemize}
    \item $L(G)=$\pause $\{a\}^*\{b\}^*$
    \pause
    \item $L(G)^*=\{a,b\}^*$
    \pause
    \item $G'=(\{S\},\{a,b\},S,\{S\rightarrow aS|bS|\epsilon\})$
  \end{itemize}
\end{frame}


\section{Vollständige Induktion}

\begin{frame}{Prinzip der vollständigen Induktion}
  Zu beweisen ist $\forall n\in \mathbb{N}_0:A(n)$. Man zeigt:
  \begin{itemize}
    \item Für ein festes, aber beliebiges n gilt: $A(n)\Rightarrow A(n+1)$
    \item A(0) ist wahr
    \item Gezeigt wurde: $A(0)\Rightarrow A(1) \Rightarrow A(2) \Rightarrow ...$
  \end{itemize}
\end{frame}

\begin{frame}{Vollständige Induktion}
  Mit der Definition
  \begin{align*}
    x_0     &= 0 \\
    \forall n\in\mathbb{N}_0: x_{n+1} &= x_n + 2 \\
  \end{align*}
  kann man die Hypothese
  $
    \forall n\in\mathbb{N}_0:  x_n = 2n
  $
  beweisen.
\end{frame}

\begin{frame}{Vollständige Induktion}
  \textbf{Induktionsanfang}
  \begin{itemize}
    \item Zu zeigen: $x_n=2n$ für n=0
    \item $x_0=0$ nach beiden Definitionen
  \end{itemize}
\end{frame}

\begin{frame}{Vollständige Induktion}
  \textbf{Induktionsvorraussetzung}
  \begin{itemize}
    \item Für ein beliebiges aber festes $n$ gilt:  $x_n = 2n$
    \item Wichtig: n ist nicht variabel, die Induktionsvorraussetzung gilt nicht für alle n!
  \end{itemize}
\end{frame}

\begin{frame}{Vollständige Induktion}
  \textbf{Induktionsschluss}
  \begin{itemize}
    \item Zeige: Für das beliebige aber feste $n$ gilt: $x_{n+1} = 2(n+1)$
    \item Beweis:  
      \begin{align*}
        x_{n+1} &= x_n + 2 &&\text{ nach Definition}\\
        &= 2n  + 2    && \text{ nach Induktionsvoraussetzung} \\
        &= 2(n+1) && \text{ fertig}\\
      \end{align*}
  \end{itemize}
\end{frame}

\begin{frame}{Klausuraufgabe: SS 2014 A5}
  Gegeben sei eine natürliche Zahl $a\in \mathbb{N}_+$. Die Abbildung $S:\mathbb{N}_0\rightarrow \mathbb{Z}$ sei induktiv definiert durch 
  \begin{align*}
    S(0)&=1,\\
    \forall k\in\mathbb{N}_0:S(k+1)&=a^{k+1}+S(k).
  \end{align*}
  Beweisen Sie durch vollständige Induktion, dass gilt:
  \begin{align*}
    \forall k\in\mathbb{N}_0:(a-1)S(k)=a^{k+1}-1.
  \end{align*}
\end{frame}

\begin{frame}[t]{Klausuraufgabe: SS 2014 A5}
  \begin{align*}
    S(0)&=1\\
    \forall k\in\mathbb{N}_0:S(k+1)&=a^{k+1}+S(k)\\
    \text{zu zeigen }\forall k\in\mathbb{N}_0:(a-1)S(k)&=a^{k+1}-1
  \end{align*}
  \pause
  \textbf{Induktionsanfang}\\
  $k=0$: Dann ist $(a-1)S(0)=a-1=a^{0+1}-1$
\end{frame}

\begin{frame}[t]{Klausuraufgabe: SS 2014 A5}
  \begin{align*}
    S(0)&=1\\
    \forall k\in\mathbb{N}_0:S(k+1)&=a^{k+1}+S(k)\\
    \text{zu zeigen }\forall k\in\mathbb{N}_0:(a-1)S(k)&=a^{k+1}-1
  \end{align*}
  \textbf{Induktionsvoraussetzung}\\
  für ein beliebiges aber festes $k$ gelte: $(a-1)S(k)=a^{k+1}-1$
\end{frame}

\begin{frame}[t]{Klausuraufgabe: SS 2014 A5}
  \begin{align*}
    S(0)&=1\\
    \forall k\in\mathbb{N}_0:S(k+1)&=a^{k+1}+S(k)\\
    \text{zu zeigen }\forall k\in\mathbb{N}_0:(a-1)S(k)&=a^{k+1}-1
  \end{align*}
  \textbf{Induktionsschluss $k\rightarrow k+1$}\\
  zu zeigen: $(a-1)S(k+1)=a^{(k+1)+1}-1$
  \pause
  \begin{align*}
    \action<+->{(a-1)S(k+1)&=(a-1)(a^{k+1}+S(k)) &&\text{ nach Definition}\\}
    \action<+->{&= (a-1)a^{k+1}+(a-1)S(k)\\}
    \action<+->{&= (a-1)a^{k+1}+a^{k+1}-1 && \text{ nach I.V.}\\}
    \action<+->{&= a^{(k+1)+1}-1}
  \end{align*}
\end{frame}

\begin{frame}{Klausuraufgabe: WS 2012/13 A3}
  Gegeben sei folgende Funktion $f:\{a,b\}^*\rightarrow\{a,b\}^*$:
  \begin{align*}
    f(\epsilon)&=\epsilon\\
    \forall w\in\{a,b\}^*: f(aw)&=bf(w)\\
    \forall w\in\{a,b\}^*: f(bw)&=af(w)
  \end{align*}
  Beweisen Sie per Induktion, dass gilt:
  \begin{align*}
    \forall w_1,w_2\in\{a,b\}^*: f(w_1w_2)&=f(w_1)f(w_2)
  \end{align*}
\end{frame}

\begin{frame}[t]{Klausuraufgabe: WS 2012/13 A3}
  \begin{align*}
    f(\epsilon)&=\epsilon\\
    \forall w\in A^*: f(aw)&=bf(w)\\
    \forall w\in A^*: f(bw)&=af(w)\\
    \text{zu zeigen }\forall w=w_1w_2\in A^n: f(w_1w_2)&=f(w_1)f(w_2)
  \end{align*}
  \pause
  \textbf{Induktionsanfang}\\
  $n=0$: $\{a,b\}^0=\{\epsilon\}$\\
  \pause
  $w_1=w_2=\epsilon$\\
  \pause
  $f(\epsilon\epsilon)=f(\epsilon)=\epsilon=\epsilon\epsilon=f(\epsilon)f(\epsilon)$
\end{frame}

\begin{frame}[t]{Klausuraufgabe: WS 2012/13 A3}
  \begin{align*}
    f(\epsilon)&=\epsilon\\
    \forall w\in A^*: f(aw)&=bf(w)\\
    \forall w\in A^*: f(bw)&=af(w)\\
    \text{zu zeigen }\forall w=w_1w_2\in A^n: f(w_1w_2)&=f(w_1)f(w_2)
  \end{align*}
  \textbf{Induktionsvoraussetzung}\\
  Für alle Wörter $w'$ mit beliebiger, aber fester Länge $n\in\mathbb{N}_0$ gelte:\\
  $\forall w'\in A^*$ mit $w'=w_1w_2: f(w_1w_2)=f(w_1)f(w_2)$
\end{frame}

\begin{frame}[t]{Klausuraufgabe: WS 2012/13 A3}
  \begin{align*}
    f(\epsilon)&=\epsilon\\
    \forall w\in A^*: f(aw)&=bf(w)\\
    \forall w\in A^*: f(bw)&=af(w)\\
    \text{zu zeigen }\forall w=w_1w_2\in A^n: f(w_1w_2)&=f(w_1)f(w_2)
  \end{align*}
  \textbf{Induktionsschritt} beliebiges $w\in A^{n+1}$\\
  \pause
  \begin{itemize}
    \item $w=aw'$:
    \pause
    $f(w)=f(aw')=bf(w')=bf(w_1w_2)=bf(w_1)f(w_2)=f(aw_1)f(w_2)$
    \pause
    \item $w=bw'$: $f(w)=f(bw')=af(w')=af(w_1w_2)=af(w_1)f(w_2)=f(bw_1)f(w_2)$
  \end{itemize}
\end{frame}

\section{Huffman-Codierung}

\begin{frame}{Huffman-Codierung}
\begin{itemize}
\item Übersetzungsfunktion h(x) gesucht
\item $\epsilon$-freier und Präfixfreier Homomorphismus
\item $h(x):A\to \{0,1\}^*$
\item h(w) soll dabei möglichst kurz sein.
\item Zeichen, die häufiger vorkommen, bekommen einen kürzeren Code.
\end{itemize}
\end{frame}

\begin{frame}
\begin{center}
w=aafbcdfbfbeefbcfbbfeb
    \qquad
  \begin{tabular}{ccccccc}
    \toprule
    x & a & b& c& d& e& f \\
    $N_x(w)$ &2 &7 &2 & 1& 3& 6\\
    \bottomrule
  \end{tabular}
\end{center}
\pause
\begin{center}
  \begin{tikzpicture}
    [level 1/.style={sibling distance=40mm},
    level 2/.style={sibling distance=20mm},
    level 3/.style={sibling distance=20mm}]
    \node {}%$22$}
    child{node{}
    edge from parent[draw=none]
    child { node {}
    edge from parent[draw=none]
      child {node {}
      edge from parent[draw=none]
        child {node {1,d} 
        edge from parent[draw=none]
        }
        child {node {2,c} 
        edge from parent[draw=none]
        }
      }
      child {node {2,a} 
      edge from parent[draw=none]
      }
    } 
    child{node{e,3}edge from parent[draw=none]}
    }
    child{node{}
    edge from parent[draw=none]
    child{node{f,6}edge from parent[draw=none]}
    child{node{b,7}edge from parent[draw=none]}
    };
  \end{tikzpicture}
\end{center}
\end{frame}

\begin{frame}
\begin{center}
w=aafbcdfbfbeefbcfbbfeb
    \qquad
  \begin{tabular}{ccccccc}
    \toprule
    x & a & b& c& d& e& f \\
    $N_x(w)$ &2 &7 &2 & 1& 3& 6\\
    \bottomrule
  \end{tabular}
\end{center}
\begin{center}
  \begin{tikzpicture}
    [level 1/.style={sibling distance=40mm},
    level 2/.style={sibling distance=20mm},
    level 3/.style={sibling distance=20mm}]
    \node {}%$22$}
    child{node{}
    edge from parent[draw=none]
    child { node {}
    edge from parent[draw=none]
      child {node {3}
      edge from parent[draw=none]
        child {node {1,d} 
             }
        child {node {2,c} 
        }
      }
      child {node {2,a} 
      edge from parent[draw=none]
      }
    } 
    child{node{e,3}edge from parent[draw=none]}
    }
    child{node{}
    edge from parent[draw=none]
    child{node{f,6}edge from parent[draw=none]}
    child{node{b,7}edge from parent[draw=none]}
    };
  \end{tikzpicture}
\end{center}
\end{frame}

\begin{frame}
\begin{center}
w=aafbcdfbfbeefbcfbbfeb
    \qquad
  \begin{tabular}{ccccccc}
    \toprule
    x & a & b& c& d& e& f \\
    $N_x(w)$ &2 &7 &2 & 1& 3& 6\\
    \bottomrule
  \end{tabular}
\end{center}
\begin{center}
  \begin{tikzpicture}
    [level 1/.style={sibling distance=40mm},
    level 2/.style={sibling distance=20mm},
    level 3/.style={sibling distance=20mm}]
    \node {}%$22$}
    child{node{}
    edge from parent[draw=none]
    child { node {5}
    edge from parent[draw=none]
      child {node {3}
        child {node {1,d} 
             }
        child {node {2,c} 
        }
      }
      child {node {2,a} 
      }
    } 
    child{node{e,3}edge from parent[draw=none]}
    }
    child{node{}
    edge from parent[draw=none]
    child{node{f,6}edge from parent[draw=none]}
    child{node{b,7}edge from parent[draw=none]}
    };
  \end{tikzpicture}
\end{center}
\end{frame}

\begin{frame}
\begin{center}
w=aafbcdfbfbeefbcfbbfeb
    \qquad
  \begin{tabular}{ccccccc}
    \toprule
    x & a & b& c& d& e& f \\
    $N_x(w)$ &2 &7 &2 & 1& 3& 6\\
    \bottomrule
  \end{tabular}
\end{center}
\begin{center}
  \begin{tikzpicture}
    [level 1/.style={sibling distance=40mm},
    level 2/.style={sibling distance=20mm},
    level 3/.style={sibling distance=20mm}]
    \node {}%$22$}
    child{node{8}
    edge from parent[draw=none]
    child { node {5}
      child {node {3}
        child {node {1,d} 
             }
        child {node {2,c} 
        }
      }
      child {node {2,a} 
      }
    } 
    child{node{e,3}}
    }
    child{node{}
    edge from parent[draw=none]
    child{node{f,6}edge from parent[draw=none]}
    child{node{b,7}edge from parent[draw=none]}
    };
  \end{tikzpicture}
\end{center}
\end{frame}

\begin{frame}
\begin{center}
w=aafbcdfbfbeefbcfbbfeb
    \qquad
  \begin{tabular}{ccccccc}
    \toprule
    x & a & b& c& d& e& f \\
    $N_x(w)$ &2 &7 &2 & 1& 3& 6\\
    \bottomrule
  \end{tabular}
\end{center}
\begin{center}
  \begin{tikzpicture}
    [level 1/.style={sibling distance=40mm},
    level 2/.style={sibling distance=20mm},
    level 3/.style={sibling distance=20mm}]
    \node {}%$22$}
    child{node{8}
    edge from parent[draw=none]
    child { node {5}
      child {node {3}
        child {node {1,d} 
             }
        child {node {2,c} 
        }
      }
      child {node {2,a} 
      }
    } 
    child{node{e,3}}
    }
    child{node{13}
    edge from parent[draw=none]
    child{node{f,6}}
    child{node{b,7}}
    };
  \end{tikzpicture}
\end{center}
\end{frame}

\begin{frame}
\begin{center}
w=aafbcdfbfbeefbcfbbfeb
    \qquad
  \begin{tabular}{ccccccc}
    \toprule
    x & a & b& c& d& e& f \\
    $N_x(w)$ &2 &7 &2 & 1& 3& 6\\
    \bottomrule
  \end{tabular}
\end{center}
\begin{center}
  \begin{tikzpicture}
    [level 1/.style={sibling distance=40mm},
    level 2/.style={sibling distance=20mm},
    level 3/.style={sibling distance=20mm}]
    \node {21}%$22$}
    child{node{8}
    child { node {5}
      child {node {3}
        child {node {1,d} 
             }
        child {node {2,c} 
        }
      }
      child {node {2,a} 
      }
    } 
    child{node{e,3}}
    }
    child{node{13}
    child{node{f,6}}
    child{node{b,7}}
    };
  \end{tikzpicture}
\end{center}
\end{frame}

\begin{frame}
\begin{center}
w=aafbcdfbfbeefbcfbbfeb
    \qquad
  \begin{tabular}{ccccccc}
    \toprule
    x & a & b& c& d& e& f \\
    $N_x(w)$ &2 &7 &2 & 1& 3& 6\\
    \bottomrule
  \end{tabular}
\end{center}
\begin{center}
  \begin{tikzpicture}
    [level 1/.style={sibling distance=40mm},
    level 2/.style={sibling distance=20mm},
    level 3/.style={sibling distance=20mm}]
    \node {21}%$22$}
    child{node{8}
    child { node {5}
      child {node {3}
        child {node {1,d} 
        edge from parent node[left] {0}
             }
        child {node {2,c} edge from parent node[right] {1}
        }
        edge from parent node[left] {0}
      }
      child {node {2,a} edge from parent node[right] {1}
      }
      edge from parent node[left] {0}
    } 
    child{node{e,3}edge from parent node[right] {1}}
    edge from parent node[left] {0}
    }
    child{node{13}
    child{node{f,6}edge from parent node[left] {0}}
    child{node{b,7}edge from parent node[right] {1}}
    edge from parent node[right] {1}
    };
  \end{tikzpicture}
\end{center}
\end{frame}

\begin{frame}
\begin{center}
  \begin{tikzpicture}
    [level 1/.style={sibling distance=40mm},
    level 2/.style={sibling distance=20mm},
    level 3/.style={sibling distance=20mm}]
    \node {21}%$22$}
    child{node{8}
    child { node {5}
      child {node {3}
        child {node {1,d} 
        edge from parent node[left] {0}
             }
        child {node {2,c} edge from parent node[right] {1}
        }
        edge from parent node[left] {0}
      }
      child {node {2,a} edge from parent node[right] {1}
      }
      edge from parent node[left] {0}
    } 
    child{node{e,3}edge from parent node[right] {1}}
    edge from parent node[left] {0}
    }
    child{node{13}
    child{node{f,6}edge from parent node[left] {0}}
    child{node{b,7}edge from parent node[right] {1}}
    edge from parent node[right] {1}
    };
  \end{tikzpicture}
  \end{center}
   \begin{tabular}{ccccccc}
    \toprule
    x & a & b& c& d& e& f \\
    $h(x)$ &001 &11 &0001 & 0000& 01& 10\\
    \bottomrule
  \end{tabular}
\end{frame}

\begin{frame}{Huffman-Codierung}
\begin{itemize}
\item w=aafbcdfbfbeefbcfbbfeb
\item h(w)=00100110110001000010111011010110110001101111100111
\item Die Codierung von w ist 50 Zeichen lang.
\end{itemize}
\end{frame}

\begin{frame}{Klausuraufgabe: SS 2011 A4}
Gegeben sei das Alphabet $A=\{a,b,c,d,e,f,g\}$ und ein Wort $w\in A^*$ in dem die Symbole mit folgenden Häufigkeiten vorkommen:
\begin{center}
\begin{tabular}{ccccccc}
    \toprule
    a & b& c& d& e& f &g\\
    \midrule
    11 &3 &11 & 24& 8& 7&36\\
    \bottomrule
\end{tabular}
\end{center}
\begin{itemize}
\item Zeichnen Sie den Huffman-Baum.
\item Geben Sie die Huffman-Codierung des Wortes $bad$ an.
\end{itemize}
\end{frame}

\begin{frame}{Klausuraufgabe: SS 2011 A4}
\begin{center}
  \begin{tikzpicture}
    [level 1/.style={sibling distance=45mm},
    level 2/.style={sibling distance=30mm},
    level 3/.style={sibling distance=15mm}]
    \node {}
    child{node{}
    child { node {}
      child {node {}
        child {node {3,b} 
        edge from parent[draw=none]
             }
        child {node {7,f} edge from parent[draw=none]
        }
        edge from parent[draw=none]
      }
      child {node {8,e} edge from parent[draw=none]
      }
      edge from parent[draw=none]
    } 
    child{node{}
    	child{node {11,a}edge from parent[draw=none]}
	child{node {11,c}edge from parent[draw=none]}
    edge from parent[draw=none]}
    edge from parent[draw=none]
    }
    child{node{}
    child{node{24,d}edge from parent[draw=none]}
    child{node{36,g}edge from parent[draw=none]}
    edge from parent[draw=none]
    };
  \end{tikzpicture}
  \end{center}
\end{frame}

\begin{frame}{Klausuraufgabe: SS 2011 A4}
\begin{center}
  \begin{tikzpicture}
    [level 1/.style={sibling distance=45mm},
    level 2/.style={sibling distance=30mm},
    level 3/.style={sibling distance=15mm}]
    \node {}
    child{node{}
    child { node {}
      child {node {10}
        child {node {3,b} 
        edge from parent node[left] {0}
             }
        child {node {7,f} edge from parent node[right] {1}
        }
        edge from parent[draw=none]
      }
      child {node {8,e} edge from parent[draw=none]
      }
      edge from parent[draw=none]
    } 
    child{node{}
    	child{node {11,a}edge from parent[draw=none]}
	child{node {11,c}edge from parent[draw=none]}
    edge from parent[draw=none]}
    edge from parent[draw=none]
    }
    child{node{}
    child{node{24,d}edge from parent[draw=none]}
    child{node{36,g}edge from parent[draw=none]}
    edge from parent[draw=none]
    };
  \end{tikzpicture}
  \end{center}
\end{frame}

\begin{frame}{Klausuraufgabe: SS 2011 A4}
\begin{center}
  \begin{tikzpicture}
    [level 1/.style={sibling distance=45mm},
    level 2/.style={sibling distance=30mm},
    level 3/.style={sibling distance=15mm}]
    \node {}
    child{node{}
    child { node {18}
      child {node {10}
        child {node {3,b} 
        edge from parent node[left] {0}
             }
        child {node {7,f} edge from parent node[right] {1}
        }
        edge from parent node[left] {0}
      }
      child {node {8,e} edge from parent node[right] {1}
      }
      edge from parent[draw=none]
    } 
    child{node{}
    	child{node {11,a}edge from parent[draw=none]}
	child{node {11,c}edge from parent[draw=none]}
    edge from parent[draw=none]}
    edge from parent[draw=none]
    }
    child{node{}
    child{node{24,d}edge from parent[draw=none]}
    child{node{36,g}edge from parent[draw=none]}
    edge from parent[draw=none]
    };
  \end{tikzpicture}
  \end{center}
\end{frame}

\begin{frame}{Klausuraufgabe: SS 2011 A4}
\begin{center}
  \begin{tikzpicture}
    [level 1/.style={sibling distance=45mm},
    level 2/.style={sibling distance=30mm},
    level 3/.style={sibling distance=15mm}]
    \node {}
    child{node{}
    child { node {18}
      child {node {10}
        child {node {3,b} 
        edge from parent node[left] {0}
             }
        child {node {7,f} edge from parent node[right] {1}
        }
        edge from parent node[left] {0}
      }
      child {node {8,e} edge from parent node[right] {1}
      }
      edge from parent[draw=none]
    } 
    child{node{22}
    	child{node {11,a}edge from parent node[left] {0}}
	child{node {11,c}edge from parent node[right] {1}}
    edge from parent[draw=none]}
    edge from parent[draw=none]
    }
    child{node{}
    child{node{24,d}edge from parent[draw=none]}
    child{node{36,g}edge from parent[draw=none]}
    edge from parent[draw=none]
    };
  \end{tikzpicture}
  \end{center}
\end{frame}

\begin{frame}{Klausuraufgabe: SS 2011 A4}
\begin{center}
  \begin{tikzpicture}
    [level 1/.style={sibling distance=45mm},
    level 2/.style={sibling distance=30mm},
    level 3/.style={sibling distance=15mm}]
    \node {}
    child{node{40}
    child { node {18}
      child {node {10}
        child {node {3,b} 
        edge from parent node[left] {0}
             }
        child {node {7,f} edge from parent node[right] {1}
        }
        edge from parent node[left] {0}
      }
      child {node {8,e} edge from parent node[right] {1}
      }
      edge from parent node[left] {0}
    } 
    child{node{22}
    	child{node {11,a}edge from parent node[left] {0}}
	child{node {11,c}edge from parent node[right] {1}}
    edge from parent node[right] {1}}
    edge from parent[draw=none]
    }
    child{node{}
    child{node{24,d}edge from parent[draw=none]}
    child{node{36,g}edge from parent[draw=none]}
    edge from parent[draw=none]
    };
  \end{tikzpicture}
  \end{center}
\end{frame}

\begin{frame}{Klausuraufgabe: SS 2011 A4}
\begin{center}
  \begin{tikzpicture}
    [level 1/.style={sibling distance=45mm},
    level 2/.style={sibling distance=30mm},
    level 3/.style={sibling distance=15mm}]
    \node {}
    child{node{40}
    child { node {18}
      child {node {10}
        child {node {3,b} 
        edge from parent node[left] {0}
             }
        child {node {7,f} edge from parent node[right] {1}
        }
        edge from parent node[left] {0}
      }
      child {node {8,e} edge from parent node[right] {1}
      }
      edge from parent node[left] {0}
    } 
    child{node{22}
    	child{node {11,a}edge from parent node[left] {0}}
	child{node {11,c}edge from parent node[right] {1}}
    edge from parent node[right] {1}}
    edge from parent[draw=none]
    }
    child{node{60}
    child{node{24,d}edge from parent node[left] {0}}
    child{node{36,g}edge from parent node[right] {1}}
    edge from parent[draw=none]
    };
  \end{tikzpicture}
  \end{center}
\end{frame}

\begin{frame}{Klausuraufgabe: SS 2011 A4}
\begin{center}
  \begin{tikzpicture}
    [level 1/.style={sibling distance=45mm},
    level 2/.style={sibling distance=30mm},
    level 3/.style={sibling distance=15mm},yscale=0.9]
    \node {100}
    child{node{40}
    child { node {18}
      child {node {10}
        child {node {3,b} 
        edge from parent node[left] {0}
             }
        child {node {7,f} edge from parent node[right] {1}
        }
        edge from parent node[left] {0}
      }
      child {node {8,e} edge from parent node[right] {1}
      }
      edge from parent node[left] {0}
    } 
    child{node{22}
    	child{node {11,a}edge from parent node[left] {0}}
	child{node {11,c}edge from parent node[right] {1}}
    edge from parent node[right] {1}}
    edge from parent node[left] {0}
    }
    child{node{60}
    child{node{24,d}edge from parent node[left] {0}}
    child{node{36,g}edge from parent node[right] {1}}
    edge from parent node[right] {1}
    };
  \end{tikzpicture}
  \end{center}
  $h(bad)=0000 \text{ } 010\text{ } 10$
\end{frame}

\begin{frame}{Klausuraufgabe: SS 2011 A4}
Gegeben seien zwei Codierungen über dem Alphabet $A=\{a,b,c,d,e\}$
\begin{center}
\begin{tabular}{cccccc}
    \toprule
    x & a & b& c& d& e \\
    $h(x)$ &00 &10 &11 & 010& 011\\
    \bottomrule
  \end{tabular}
  \end{center}
  \begin{center}
  \begin{tabular}{cccccc}
    \toprule
    x & a & b& c& d& e \\
    $h(x)$ &10 &11 &001 & 010& 011\\
    \bottomrule
  \end{tabular}
  \end{center}
  Welche der beiden Codierungen ist eine gültige Huffman-Codierung?
\end{frame}

\begin{frame}
\begin{center}
  \begin{tabular}{cccccc}
    \toprule
    x & a & b& c& d& e \\
    $h(x)$ &00 &10 &11 & 010& 011\\
    \bottomrule
  \end{tabular}
\end{center}
\begin{center}
  \begin{tikzpicture}
    [level 1/.style={sibling distance=40mm},
    level 2/.style={sibling distance=20mm},
    level 3/.style={sibling distance=20mm}]
    \node {}%$22$}
    child{node{}
    child { node {a}
      edge from parent node[left] {0}
    } 
    child{node{}
    child{ node {d}edge from parent node[left] {0}}
    child{ node {e}edge from parent node[right] {1}}
    edge from parent node[right] {1}}
    edge from parent node[left] {0}
    }
    child{node{}
    child{node{b}edge from parent node[left] {0}}
    child{node{c}edge from parent node[right] {1}}
    edge from parent node[right] {1}
    };
  \end{tikzpicture}
\end{center}
\end{frame}

\begin{frame}
\begin{center}
  \begin{tabular}{cccccc}
    \toprule
    x & a & b& c& d& e \\
    $h(x)$ &10 &11 &001 & 010& 011\\
    \bottomrule
  \end{tabular}
\end{center}
\begin{center}
  \begin{tikzpicture}
    [level 1/.style={sibling distance=40mm},
    level 2/.style={sibling distance=20mm},
    level 3/.style={sibling distance=10mm}]
    \node {}%$22$}
    child{node{}
    child { node {}
    child{node{}edge from parent[draw=none]}
    child{node{c}edge from parent node[right] {1}}
      edge from parent node[left] {0}
    } 
    child{node{}
    child{ node {d}edge from parent node[left] {0}}
    child{ node {e}edge from parent node[right] {1}}
    edge from parent node[right] {1}}
    edge from parent node[left] {0}
    }
    child{node{}
    child{node{a}edge from parent node[left] {0}}
    child{node{b}edge from parent node[right] {1}}
    edge from parent node[right] {1}
    };
  \end{tikzpicture}
\end{center}
\end{frame}


\end{document}
