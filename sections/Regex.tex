\section{Reguläre Ausdrücke / Endliche Akzeptoren}

\begin{frame}{Reguläre Ausdrücke}
  $A$ ist ein beliebiges Alphabet, $Z=\{|,(,),*,\emptyset\}$ \\
  Dann:
  \begin{itemize}
    \item $\emptyset$ ist ein regul\"arer Ausdruck
    \item F\"ur jedes $a \in A$ ist $a$ ein regul\"arer Ausdruck
    \item F\"ur die regul\"aren Ausdr\"ucke $A_{1}$ und $A_{2}$ sind $(A_{1}|A_{2})$ und $(A_{1}A_{2})$ auch regul\"are Ausdr\"ucke
    \item Ist $A$ ein regul\"arer Ausdruck, so ist auch $A*$
    \item Nichts anderes ist ein regul\"arer Ausdruck
  \end{itemize}
\end{frame}

\begin{frame}{Reguläre Ausdrücke}
  $\langle R \rangle$ sei die Sprache über dem regulären Ausdruck $R$. \\
  Dann:
  \begin{itemize}
    \item $\forall x\in A: \langle x \rangle = \{x\}$
    \item $\langle R_1 | R_2 \rangle = \langle R_1 \rangle \cup \langle R_2 \rangle$
    \item $\langle R_1R_2\rangle = \langle R_1 \rangle \langle R_2 \rangle$
    \item $\langle R*\rangle = \langle R \rangle ^*$
  \end{itemize}
\end{frame}

\begin{frame}{Reguläre Ausdrücke}
  \begin{itemize}
    \item $R=a$, $\langle R\rangle = \{a\}$
    \pause
    \item $R=(a|b)$, $\langle R\rangle = \{a\}\cup\{b\}= \{a,b\}$
    \pause
    \item $R=(aa|b)*$, $\langle R\rangle = \{aa,b\}^*$
    \pause
    \item $R=aba(ab|aa)*$, $\langle R\rangle = \{aba\}\{ab,aa\}^*$
    \pause
    \item $R=aa|(bb)*$, $\langle R\rangle = \{aa\}\cup\{bb\}^*$
    \pause
    \item $R=\emptyset$, $\langle R\rangle = \{\}$
    \pause
    \item $R=\emptyset *$, $\langle R\rangle = \{\epsilon\}$
    \pause
    \item  $R=\emptyset *|ab$, $\langle R\rangle = \{\epsilon ,ab\}$
  \end{itemize}
\end{frame}

\begin{frame}{Endliche Akzeptoren}
  Ein endlicher Akzeptor $A = (Z, z_{0}, X, f, F)$ wird festgelegt durch
  \begin{itemize}
    \item endliche Zustandsmenge $Z$
    \item einen Anfangszustand $z_{0} \in Z$
    \item ein Eingabealphabet $X$
    \item eine Zustands\"uberf\"uhrungsfunktion $f: Z \times X \rightarrow Z$
    \item eine Menge $F \subseteq Z$ akzeptierender Zust\"ande
  \end{itemize}					
\end{frame}

\begin{frame}{Endliche Akzeptoren}
  \begin{center}
    \begin{tikzpicture}[shorten >=1pt,node distance=2.5cm,auto,initial text=,>=stealth]

	\node[state, initial, accepting] (0) {0};
	\node[state, accepting] [right of = 0] (1) {1};
	\node[state] [right of = 1] (2) {2};
	
	\path[->]
		(0) edge [loop above] node [above] {a} ()
			edge [bend right] node [below] {b} (1)
		(1) edge [bend right] node [above] {a} (0)
			edge node [above] {b} (2)
		(2) edge [loop right] node [right] {a,b} ();		
	
    \end{tikzpicture}
  \end{center}
  \pause
  \begin{align*}
    L=\{a,b\}^*\setminus \{w_1bbw_2|w_1,w_2\in \{a,b\}^*\}
  \end{align*}
\end{frame}

\begin{frame}{Klausuraufgabe: WS 2013/14 A6}
Es sei der folgende endliche Akzeptor $M$ mit Zustandsmenge $Z=\{A,B,C,D\}$ und Eingabealphabet $X=\{a,b\}$ gegeben:
  \begin{center}
    \begin{tikzpicture}[shorten >=1pt,node distance=2cm,auto,initial text=,>=stealth]

	\node[state, initial] (0) {A};
	\node[state, accepting] [right of = 0] (1) {B};
	\node[state, accepting] [right of = 1] (2) {C};
	\node[state] [right of = 2] (3) {D};
	
	\path[->]
		(0) edge [loop above] node [above] {a} ()
			edge node [above] {$b$} (1)
		(1) edge [loop above] node [above] {b} ()
			edge node [above] {$a$} (2)
		(2) edge [loop above] node [above] {a} ()
			edge node [above] {$b$} (3)
		(3) edge [loop above] node [above] {a,b} ();		
	
    \end{tikzpicture}
  \end{center}
  \begin{itemize}
    \item Geben Sie einen regulären Ausdruck $R$ an, so dass gilt: $\langle R\rangle = L(M)$.
    \item Geben sie einen endlichen Akzeptor an, der folgende formale Sprache akzeptiert:
      \begin{align*}
        \{w|w\in L(M)\text{ und die Anzahl }N_a(w)\text{ der }a\text{ in }w\text{ ist gerade}\}
      \end{align*}
  \end{itemize}
\end{frame}

\begin{frame}{Klausuraufgabe: WS 2013/14 A6}
  \begin{center}
    \begin{tikzpicture}[shorten >=1pt,node distance=2cm,auto,initial text=,>=stealth]

	\node[state, initial] (0) {A};
	\node[state, accepting] [right of = 0] (1) {B};
	\node[state, accepting] [right of = 1] (2) {C};
	\node[state] [right of = 2] (3) {D};
	
	\path[->]
		(0) edge [loop above] node [above] {a} ()
			edge node [above] {$b$} (1)
		(1) edge [loop above] node [above] {b} ()
			edge node [above] {$a$} (2)
		(2) edge [loop above] node [above] {a} ()
			edge node [above] {$b$} (3)
		(3) edge [loop above] node [above] {a,b} ();		
	
    \end{tikzpicture}
  \end{center}
  Geben Sie einen regulären Ausdruck $R$ an, so dass gilt: $\langle R\rangle = L(M)$.
  \pause
  \begin{itemize}
    \item $R_B=a*bb*$
    \pause
    \item $R_C=a*bb*aa*$
    \pause
    \item $R=R_B|R_C=(a*bb*)|(a*bb*aa*)=a*bb*a*$
  \end{itemize}
\end{frame}

\begin{frame}{Klausuraufgabe: WS 2013/14 A6}
  \begin{align*}
    \{w|w\in L(M)\text{ und die Anzahl }N_a(w)\text{ der }a\text{ in }w\text{ ist gerade}\}
  \end{align*}
  \begin{center}
    \begin{tikzpicture}[shorten >=1pt,node distance=2cm,auto,initial text=,>=stealth]

	\node[state, initial] (0) {A};
	\node[state, accepting] [right of = 0] (1) {B};
	\node[state, accepting] [right of = 1] (2) {C};
	\node[state] [right of = 2] (3) {D};
	
	\path[->]
		(0) edge [loop above] node [above] {a} ()
			edge node [above] {$b$} (1)
		(1) edge [loop above] node [above] {b} ()
			edge node [above] {$a$} (2)
		(2) edge [loop above] node [above] {a} ()
			edge node [above] {$b$} (3)
		(3) edge [loop above] node [above] {a,b} ();		
	
    \end{tikzpicture}
  \end{center}
\end{frame}

\begin{frame}{Klausuraufgabe: WS 2013/14 A6}
  \begin{align*}
    \{w|w\in L(M)\text{ und die Anzahl }N_a(w)\text{ der }a\text{ in }w\text{ ist gerade}\}
  \end{align*}
  \begin{center}
    \begin{tikzpicture}[shorten >=1pt,node distance=2cm,auto,initial text=,>=stealth]

	\node[state, initial] (0) {A};
	\node[state, accepting] [right of = 0] (1) {B};
	\node[state, accepting] [right of = 1] (2) {C};
	\node[state] [right of = 2] (3) {D};
	\node[state] [below of = 0] (4) {A'};
	\node[state] [below of = 1] (5) {B'};
	\node[state] [below of = 2] (6) {C'};
	
	\path[->]
		(0) edge [loop above] node [above] {a} ()
			edge node [above] {$b$} (1)
		(1) edge [loop above] node [above] {b} ()
			edge node [above] {$a$} (2)
		(2) edge [loop above] node [above] {a} ()
			edge node [above] {$b$} (3)
		(3) edge [loop above] node [above] {a,b} ();		
	
    \end{tikzpicture}
  \end{center}
\end{frame}

\begin{frame}{Klausuraufgabe: WS 2013/14 A6}
  \begin{align*}
    \{w|w\in L(M)\text{ und die Anzahl }N_a(w)\text{ der }a\text{ in }w\text{ ist gerade}\}
  \end{align*}
  \begin{center}
    \begin{tikzpicture}[shorten >=1pt,node distance=2cm,auto,initial text=,>=stealth]

	\node[state, initial] (0) {A};
	\node[state, accepting] [right of = 0] (1) {B};
	\node[state, accepting] [right of = 1] (2) {C};
	\node[state] [right of = 2] (3) {D};
	\node[state] [below of = 0] (4) {A'};
	\node[state] [below of = 1] (5) {B'};
	\node[state] [below of = 2] (6) {C'};
	
	\path[->]
		(0) edge [bend left] node [right] {a} (4)
			edge node [above] {$b$} (1)
		(1) edge [loop above] node [above] {b} ()
			edge node [above] {$a$} (2)
		(2) edge [loop above] node [above] {a} ()
			edge node [above] {$b$} (3)
		(3) edge [loop above] node [above] {a,b} ()
		(4) edge [bend left] node [left] {a} (0);		
	
    \end{tikzpicture}
  \end{center}
\end{frame}

\begin{frame}{Klausuraufgabe: WS 2013/14 A6}
  \begin{align*}
    \{w|w\in L(M)\text{ und die Anzahl }N_a(w)\text{ der }a\text{ in }w\text{ ist gerade}\}
  \end{align*}
  \begin{center}
    \begin{tikzpicture}[shorten >=1pt,node distance=2cm,auto,initial text=,>=stealth]

	\node[state, initial] (0) {A};
	\node[state, accepting] [right of = 0] (1) {B};
	\node[state, accepting] [right of = 1] (2) {C};
	\node[state] [right of = 2] (3) {D};
	\node[state] [below of = 0] (4) {A'};
	\node[state] [below of = 1] (5) {B'};
	\node[state] [below of = 2] (6) {C'};
	
	\path[->]
		(0) edge [bend left] node [right] {a} (4)
			edge node [above] {$b$} (1)
		(1) edge [loop above] node [above] {b} ()
			edge node [above] {$a$} (2)
		(2) edge [loop above] node [above] {a} ()
			edge node [above] {$b$} (3)
		(3) edge [loop above] node [above] {a,b} ()
		(4) edge [bend left] node [left] {a} (0)
			edge node [above] {$b$} (5)
		(5) edge [loop below] node [below] {b} ();		
	
    \end{tikzpicture}
  \end{center}
\end{frame}

\begin{frame}{Klausuraufgabe: WS 2013/14 A6}
  \begin{align*}
    \{w|w\in L(M)\text{ und die Anzahl }N_a(w)\text{ der }a\text{ in }w\text{ ist gerade}\}
  \end{align*}
  \begin{center}
    \begin{tikzpicture}[shorten >=1pt,node distance=2cm,auto,initial text=,>=stealth]

	\node[state, initial] (0) {A};
	\node[state, accepting] [right of = 0] (1) {B};
	\node[state, accepting] [right of = 1] (2) {C};
	\node[state] [right of = 2] (3) {D};
	\node[state] [below of = 0] (4) {A'};
	\node[state] [below of = 1] (5) {B'};
	\node[state] [below of = 2] (6) {C'};
	
	\path[->]
		(0) edge [bend left] node [right] {a} (4)
			edge node [above] {$b$} (1)
		(1) edge [loop above] node [above] {b} ()
			edge node [above] {$a$} (6)
		(2) edge [loop above] node [above] {a} ()
			edge node [above] {$b$} (3)
		(3) edge [loop above] node [above] {a,b} ()
		(4) edge [bend left] node [left] {a} (0)
			edge node [above] {$b$} (5)
		(5) edge [loop below] node [below] {b} ()
			edge node [below] {$a$} (2);		
	
    \end{tikzpicture}
  \end{center}
\end{frame}

\begin{frame}{Klausuraufgabe: WS 2013/14 A6}
  \begin{align*}
    \{w|w\in L(M)\text{ und die Anzahl }N_a(w)\text{ der }a\text{ in }w\text{ ist gerade}\}
  \end{align*}
  \begin{center}
    \begin{tikzpicture}[shorten >=1pt,node distance=2cm,auto,initial text=,>=stealth]

	\node[state, initial] (0) {A};
	\node[state, accepting] [right of = 0] (1) {B};
	\node[state, accepting] [right of = 1] (2) {C};
	\node[state] [right of = 2] (3) {D};
	\node[state] [below of = 0] (4) {A'};
	\node[state] [below of = 1] (5) {B'};
	\node[state] [below of = 2] (6) {C'};
	
	\path[->]
		(0) edge [bend left] node [right] {a} (4)
			edge node [above] {$b$} (1)
		(1) edge [loop above] node [above] {b} ()
			edge node [above] {$a$} (6)
		(2) edge [bend left] node [right] {a} (6)
			edge node [above] {$b$} (3)
		(3) edge [loop above] node [above] {a,b} ()
		(4) edge [bend left] node [left] {a} (0)
			edge node [above] {$b$} (5)
		(5) edge [loop below] node [below] {b} ()
			edge node [below] {$a$} (2)
		(6) edge [bend left] node [left] {a} (2)
			edge node [below] {$b$} (3);		
	
    \end{tikzpicture}
  \end{center}
\end{frame}

\begin{frame}{Klausuraufgabe: WS 2013/14 A6}
  \begin{center}
    \begin{tikzpicture}[shorten >=1pt,node distance=2cm,auto,initial text=,>=stealth]

	\node[state, initial] (0) {A};
	\node[state, accepting] [right of = 0] (1) {B};
	\node[state, accepting] [right of = 1] (2) {C};
	\node[state] [right of = 2] (3) {D};
	
	\path[->]
		(0) edge [loop above] node [above] {a} ()
			edge node [above] {$b$} (1)
		(1) edge [loop above] node [above] {b} ()
			edge node [above] {$a$} (2)
		(2) edge [loop above] node [above] {a} ()
			edge node [above] {$b$} (3)
		(3) edge [loop above] node [above] {a,b} ();		
	
    \end{tikzpicture}
  \end{center}
  Wie viele verschiedene formale Sprachen kann man mit endlichen Akzeptoren erkennen, deren Eingabealphabet, Zustandsmenge, Anfangszustand und Zustandsüberführungsfunktion wie bei dem oben angegebenen Akzeptor $M$ sind?\\
  \pause
  16
\end{frame}

\begin{frame}{Klausuraufgabe: SS 2015 A6}
  \begin{itemize}
    \item Geben Sie einen endlichen Akzeptor an, der die formale Sprache erkennt, die durch den regulären Ausdruck $(ab)*(aa)*$ beschrieben wird.
    \item Geben Sie eine kontextfreie Grammatik an, die die formale Sprache
      \begin{align*}
        L=\{a^kb^{m+k}c^{m+l}d^l|k,l,m\in\mathbb{N}_0\}
      \end{align*}
    \item Gibt es einen regulären Ausdruck, der die formale Sprache beschreibt?
  \end{itemize}
\end{frame}

\begin{frame}{Klausuraufgabe: SS 2015 A6}
  \begin{align*}
    (ab)*(aa)*
  \end{align*}
  \begin{center}
    \begin{tikzpicture}[shorten >=1pt,node distance=2cm,auto,initial text=,>=stealth]

	\node[state, initial, accepting] (0) {A};
	\node[state] [right of = 0] (1) {B};
	\node[state, accepting] [right of = 1] (2) {C};
	\node[state] [right of = 2] (3) {D};
	\node[state] [below of = 1] (4) {E};
	
	\path[->]
		(0) edge [bend left] node [above] {a} (1)
			edge node [below] {$b$} (4)
		(1) edge [bend left] node [below] {b} (0)
			edge node [above] {$a$} (2)
		(2) edge node [below] {$b$} (4)
			edge [bend left] node [above] {a} (3)
		(3) edge [bend left] node [below]{$b$} (4)
			edge [bend left] node [below] {a} (2)
		(4) edge [loop below] node [below] {a,b} ();		
	
    \end{tikzpicture}
  \end{center}
\end{frame}

\begin{frame}{Klausuraufgabe: SS 2015 A6}
  \begin{align*}
    L=\{a^kb^{m+k}c^{m+l}d^l|k,l,m\in\mathbb{N}_0\}
  \end{align*}
  Gibt es einen regulären Ausdruck, der die formale Sprache beschreibt?\\
  \pause
  Nein.
\end{frame}
