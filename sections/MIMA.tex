\section{MIMA}

\begin{frame}{MIMA}
  \begin{itemize}
    \item Adressen: 20 bit
    \item Speicherwerte: 24 bit
    \item (die meisten) Befehle: 4 bit + 20 bit
    \item z.B. LDC 10
  \end{itemize}
\end{frame}

\begin{frame}{Beispiel: Multiplikation}
  \begin{align*}
    &\text{LDC}\; 0\\
    &\text{STV}\; prod\\
    start:\;&\text{LDC}\; 0\\
    &\text{NOT}\; \\
    &\text{ADD}\; b\\
    &\text{STV}\; b\\
    &\text{JMN}\; end\\
    &\text{LDV}\; prod\\
    &\text{ADD}\; a\\
    &\text{STV}\; prod\\
    &\text{JMP}\; start\\
    end:\;&\text{HALT}
  \end{align*}
\end{frame}

\begin{frame}{Befehle}
  \begin{itemize}
    \item \textbf{LDC const} Lädt eine 20 bit Zahl in den Akkumulator.
    \pause
    \item \textbf{LDV adr} Lädt den Speicherwert von adr in den Akkumulator.
    \pause
    \item \textbf{STV adr} Speichert den Wert vom Akkumulator in adr.
    \pause
    \item \textbf{LDIV adr} Lädt den Speicherwert vom Speicherwert von adr $M(M(adr))$ in den Akkumulator.
    \pause
    \item \textbf{STIV adr} Speichert den Wert vom Akkumulator in $M(M(adr))$.
    \pause
    \item \textbf{ADD adr} Addiert den Speicherwert von adr auf den Akkumulator und speichert das Ergebnis im Akkumulator.
  \end{itemize}
\end{frame}

\begin{frame}{Befehle}
  \begin{itemize}
    \item \textbf{AND adr} Bitweise AND vom Speicherwert von adr mit dem Akkumulator. Ergebnis wird im Akkumulator gespeichert.
    \pause
    \item \textbf{OR adr} Bitweise OR.
    \pause
    \item \textbf{XOR adr} Bitweise XOR.
    \pause
    \item \textbf{NOT} Invertiert die Bits des Akkumulators.
  \end{itemize}
\end{frame}

\begin{frame}{Befehle}
  \begin{itemize}
    \item \textbf{RAR} Rotation der Akku-Bits nach rechts. Beispiel: $101100 \rightarrow 010110$
    \pause
    \item \textbf{EQL adr} Vergleicht den Speicherwert von adr mit dem Akkumulator. Wenn die Zahlen gleich sind wird die Zweierkomplementdarstellung von -1 im Akkumulator gespeichert, wenn nicht dann wird die Zahl 0 im Akkumulator gespeichert.
    \pause
    \item \textbf{JMP adr} Programm springt an die Adresse adr und setzt mit dem Befehl in adr fort.
    \pause
    \item \textbf{JMN adr} Programm springt an die Adresse adr, falls der Akkumulator negativ ist.
  \end{itemize}
\end{frame}

\begin{frame}{Wichtige Blöcke}
  Lade -1:
  \begin{align*}
    &LDC\text{ }0\\
    &NOT
  \end{align*}
\end{frame}

\begin{frame}{Wichtige Blöcke}
  Addiere Konstante c:
  \begin{align*}
    &LDC\text{ }c\\
    &ADD\text{ }a\\
    &STV\text{ }a
  \end{align*}
\end{frame}

\begin{frame}{Wichtige Blöcke}
  Addiere negative Konstante $-c$:
  \begin{align*}
    &LDC\text{ }c-1\\
    &NOT\\
    &ADD\text{ }a\\
    &STV\text{ }a
  \end{align*}
\end{frame}

\begin{frame}{Wichtige Blöcke}
  Addiere negative Konstante $-5$:
  \begin{align*}
    &LDC\text{ }4\\
    &NOT\\
    &ADD\text{ }a\\
    &STV\text{ }a
  \end{align*}
\end{frame}

\begin{frame}{Wichtige Blöcke}
  Verwandle $a \rightarrow -a$:
  \begin{align*}
    &\text{LDV}\; a\\
    &\text{NOT}\; \\
    &\text{STV}\; a\\
    &\text{LDC}\; 1\\
    &\text{ADD}\; a\\
    &\text{STV}\; a\\
  \end{align*}
\end{frame}

\begin{frame}{Wichtige Blöcke}
  $n$ Schleifendurchläufe:
  \begin{align*}
    start: &LDC\text{ }0\\
    &NOT\\
    &ADD\text{ }n\\
    &STV\text{ }n\\
    &JMN\text{ }end\\
    &Block\\
    &JMP\text{ }start
  \end{align*}
\end{frame}

\begin{frame}{Beispiel: Division}
  \begin{minipage}[t!]{0.5\textwidth}
    \begin{align*}
      &\text{LDC}\; 0\\
      &\text{STV}\; div\\
      &\text{LDV}\; b\\
      &\text{NOT}\; \\
      &\text{STV}\; b\\
      &\text{LDC}\; 1\\
      &\text{ADD}\; b\\
      &\text{STV}\; b
    \end{align*}
  \end{minipage}
  \begin{minipage}[t!]{0.4\textwidth}
    \begin{align*}
      start:\;&\text{LDV}\; a\\
      &\text{ADD}\; b\\
      &\text{STV}\; a\\
      &\text{JMN}\; end\\
      &\text{LDC}\; 1\\
      &\text{ADD}\; div\\
      &\text{STV}\; div\\
      &\text{JMP}\; start\\
      end:\;&\text{HALT}
    \end{align*}
  \end{minipage}
\end{frame}

\begin{frame}{Aufgabenblatt: WS 2015/16 A5.2}
  Es seien $a_1$ und $a_2$ zwei verschiedene 20bit Adressen. Im Speicher stehe in Adresse $a_1$ die Zweierkomplementdarstellung einer nicht-negativen ganzen Zahl $x$, für die $2^x$ mit 24bit in Zweierkomplementdarstellung darstellbar ist. Ergänzen Sie die fehlenden Konstanten und Adressen im unvollständigen Minimalmaschinenprogramm derart, dass nach dessen Ausführung $2^x$ in Zweierkomplementdarstellung im Speicher bei Adresse $a_2$ steht. Beachten Sie, dass alle arithmetischen Ausdrücke, in denen $x$ vorkommt, keine Konstanten sind, und, dass $2^0=1$ gilt.
\end{frame}

\begin{frame}{Aufgabenblatt: WS 2015/16 A5.2}
  \begin{align*}
    &LDC\\
    &STV\\
    while: &LDC\\
    &NOT\\
    &ADD\\
    &STV\\
    &JMN \text{ } end\\
    &LDV\\
    &ADD\\
    &STV\\
    &JMP\text{ }while\\
    end: &HALT
  \end{align*}
\end{frame}

\begin{frame}{Aufgabenblatt: WS 2015/16 A5.2}
  \begin{align*}
    &LDC\text{ }1 \\
    &STV\text{ }a_2\\
    while: &LDC\\
    &NOT\\
    &ADD\\
    &STV\\
    &JMN \text{ } end\\
    &LDV\\
    &ADD\\
    &STV\\
    &JMP\text{ }while\\
    end: &HALT
  \end{align*}
\end{frame}

\begin{frame}{Aufgabenblatt: WS 2015/16 A5.2}
  \begin{align*}
    &LDC\text{ }1 \\
    &STV\text{ }a_2\\
    while: &LDC\text{ }0\\
    &NOT\\
    &ADD\text{ }a_1\\
    &STV\text{ }a_1\\
    &JMN \text{ } end\\
    &LDV\\
    &ADD\\
    &STV\\
    &JMP\text{ }while\\
    end: &HALT
  \end{align*}
\end{frame}

\begin{frame}{Aufgabenblatt: WS 2015/16 A5.2}
  \begin{align*}
    &LDC\text{ }1 \\
    &STV\text{ }a_2\\
    while: &LDC\text{ }0\\
    &NOT\\
    &ADD\text{ }a_1\\
    &STV\text{ }a_1\\
    &JMN \text{ } end\\
    &LDV\text{ }a_2\\
    &ADD\text{ }a_2\\
    &STV\text{ }a_2\\
    &JMP\text{ }while\\
    end: &HALT
  \end{align*}
\end{frame}

\begin{frame}{Aufgabenblatt: WS 2014/15 A5.2}
  Es seien $a_1$ und $a_2$ zwei verschiedene Adressen. Welche Zahlen in Zweierkomplementdarstellung stehen nach Ausführung des Programms in den Adressen $a_1$ und $a_2$ im Speicher?
  \begin{align*}
    &LDV\text{ }a_1 \\
    &XOR\text{ }a_2\\
    &STV\text{ }a_1\\
    &LDV\text{ }a_2\\
    &XOR\text{ }a_1\\
    &STV\text{ }a_2\\
    &LDV\text{ }a_1\\
    &XOR\text{ }a_2\\
    &STV\text{ }a_1
  \end{align*}
\end{frame}

\begin{frame}{Aufgabenblatt: WS 2014/15 A5.2}
  Beispiel mit zwei 8 bit Zahlen:
  \begin{align*}
    &a_1 &a_2\\
    &11101001 &00100111\\
    &11001110 &00100111\\
    &11001110 &11101001\\
    &00100111 &11101001
  \end{align*}
  \pause
  Die Speicherwerte der Adressen $a_1$ und $a_2$ werden vertauscht.
\end{frame}
