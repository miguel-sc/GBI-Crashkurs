\section{Speicher}

\begin{frame}{Menge aller Abbildungen}
Die Menge aller Abbildungen von A nach B ist definiert als $B^A$.
\begin{itemize}
\item $B^A=\{f:A\rightarrow B\}$
\item $|B^A|=|B|^{|A|}$
\item Für jedes $a\in A$ gibt es $|B|$ Möglichkeiten einen Funktionswert aus $B$ zu finden.
\item Es gibt $|A|$ Elemente in $A$, also ergibt sich für die Anzahl aller Kombinationen: $\underbrace{|B|\cdot |B| \cdots |B|}_{|A| \text{ mal}}$
\end{itemize}
\end{frame}

\begin{frame}{Funktionen als Argument einer Funktion}
Beispiel:
\begin{itemize}
\item Es sei $M$ eine Menge
\item Jede Teilmenge
  $L\subseteq M$ kann man bijektiv mit einer Abbildung $f: M\to
  \{0,1\}$, also einem $f\in \{0,1\}^M$ in Beziehung setzen.
\item z.B. $L=\{2\}\subseteq M=\{1,2,3\}$ also $f(1)=0,f(2)=1,f(3)=0$
\item Dann kann man die Vereinigung als Abbildung $V: \{0,1\}^M \times
  \{0,1\}^M\to \{0,1\}^M$ auffassen.
\end{itemize}
\end{frame}

\begin{frame}{Funktionen als Argument einer Funktion}
Beispielbild für
  $L_1=\{a,c,d\}$ und $L_2=\{b,c\}$
  
  \begin{tabular}{cccc}
      & $L_1$ & $L_2$ & $L_1\cup L_2$ \\
    x & $f_1(x)$ & $f_2(x)$ & $V(f_1,f_2)$ \\
    \midrule
    a & 1 & 0 & 1 \\
    b & 0 & 1 & 1 \\
    c & 1 & 1 & 1 \\
    d & 1 & 0 & 1 \\
    e & 0 & 0 & 0 \\
  \end{tabular}\\
Wie definiert man $V(f_1,f_2)$? \\
\pause
Zum Beispiel so:
$V(f_1,f_2) (x) = \max(f_1(x),f_2(x))$
\end{frame}


\begin{frame}{Speicher}
Ein Speicher kann als Abbildung von einer Menge aus Adressen nach einer Menge an Speicherwerten aufgefasst werden.\\
$m:\text{Adr}\to\text{Val}$\\
\begin{tabular}{cc}
    \begin{tabular}{cc}
      \toprule
      Adresse 1 & Wert 1 \\
      Adresse 2 & Wert 2 \\
      Adresse 3 & Wert 3 \\
      Adresse 4 & Wert 4 \\
       $\vdots$ & $ \vdots$ \\
      Adresse $n$ & Wert  $n$ \\
      \bottomrule \\
    \end{tabular}
    \qquad
        \begin{tabular}{cc}
      \toprule
      \text{000} & \text{10110101} \\
      \text{001} & \text{10101101} \\
      \text{010} & \text{10011101} \\
      \text{011} & \text{01110110} \\
      \text{100} & \text{00111110} \\
      \text{101} & \text{10101101} \\
      \text{110} & \text{00101011} \\
      \text{111} & \text{10101001} \\
      \bottomrule \\
    \end{tabular}
\end{tabular}
\end{frame}

\begin{frame}{Speicher}
Das Auslesen eines Wertes ist realisiert durch die memread Funktion
\begin{align*}
  \text{memread} : \text{Val}^{\text{Adr}} \times \text{Adr} &\to \text{Val} \\
             (m,a) &\mapsto m(a) \\
\end{align*}
\end{frame}

\begin{frame}{Speicher}
memwrite speichert den Wert v in Adresse a:
\begin{align*}
  \text{memwrite} : \text{Val}^{\text{Adr}} \times \text{Adr} \times \text{Val} \to& \text{Val}^{\text{Adr}} \\
             (m,a, v) \mapsto& m' 
\end{align*}
\end{frame}

\begin{frame}{Speicher}
Eigenschaften von memread und memwrite:
\begin{itemize}
\item $\text{memread}(\text{memwrite}(m,a,v),a) = v$
\item $\text{memread}(\text{memwrite}(m,a',v'),a) = \text{memread}(m,a)$
\item Auslesen einer Speicherstelle ist unabhängig davon, was vorher an einer anderen Adresse gespeichert war.
\end{itemize}
\end{frame}


\begin{frame}{Speicher}
Die Tabelle sei unser Speicher im Zustand $m\in Mem=\text{Val}^{\text{Adr}}$
    \begin{tabular}{cc}
      \toprule
      \text{000} & \text{10110101} \\
      \text{001} & \text{10101101} \\
      \text{010} & \text{10011101} \\
      \text{011} & \text{01110110} \\
      \text{100} & \text{00111110} \\
      \text{101} & \text{10101101} \\
      \text{110} & \text{00101011} \\
      \text{111} & \text{10101001} \\
      \bottomrule \\
    \end{tabular}
\begin{itemize}
\item $\text{memread}(m,100)=?$
\item $\text{memwrite}(m,110,00011000)=?$
\item $\text{memread}(\text{memwrite}(m,001,00110011),001)=?$
\item $m(101)=?$
\end{itemize}
\end{frame}

\begin{frame}{Speicher}
Die Tabelle sei unser Speicher im Zustand $m\in Mem=\text{Val}^{\text{Adr}}$
    \begin{tabular}{cc}
      \toprule
      \text{000} & \text{10110101} \\
      \text{001} & \text{10101101} \\
      \text{010} & \text{10011101} \\
      \text{011} & \text{01110110} \\
      \text{100} & \text{00111110} \\
      \text{101} & \text{10101101} \\
      \text{110} & \text{00101011} \\
      \text{111} & \text{10101001} \\
      \bottomrule \\
    \end{tabular}
\begin{itemize}
\item $\text{memread}(m,100)=00111110$
\item $\text{memwrite}(m,110,00011000)=m'$
\item $\text{memread}(\text{memwrite}(m,001,00110011),001)=00110011$
\item $m(101)=10101101$
\end{itemize}
\end{frame}