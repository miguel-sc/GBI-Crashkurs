\section{Vollständige Induktion}

\begin{frame}{Prinzip der vollständigen Induktion}
  Zu beweisen ist $\forall n\in \mathbb{N}_0:A(n)$. Man zeigt:
  \begin{itemize}
    \item Für ein festes, aber beliebiges n gilt: $A(n)\Rightarrow A(n+1)$
    \item A(0) ist wahr
    \item Gezeigt wurde: $A(0)\Rightarrow A(1) \Rightarrow A(2) \Rightarrow ...$
  \end{itemize}
\end{frame}

\begin{frame}{Vollständige Induktion}
  Mit der Definition
  \begin{align*}
    x_0     &= 0 \\
    \forall n\in\mathbb{N}_0: x_{n+1} &= x_n + 2 \\
  \end{align*}
  kann man die Hypothese
  $
    \forall n\in\mathbb{N}_0:  x_n = 2n
  $
  beweisen.
\end{frame}

\begin{frame}{Vollständige Induktion}
  \textbf{Induktionsanfang}
  \begin{itemize}
    \item Zu zeigen: $x_n=2n$ für n=0
    \item $x_0=0$ nach beiden Definitionen
  \end{itemize}
\end{frame}

\begin{frame}{Vollständige Induktion}
  \textbf{Induktionsvorraussetzung}
  \begin{itemize}
    \item Für ein beliebiges aber festes $n$ gilt:  $x_n = 2n$
    \item Wichtig: n ist nicht variabel, die Induktionsvorraussetzung gilt nicht für alle n!
  \end{itemize}
\end{frame}

\begin{frame}{Vollständige Induktion}
  \textbf{Induktionsschluss}
  \begin{itemize}
    \item Zeige: Für das beliebige aber feste $n$ gilt: $x_{n+1} = 2(n+1)$
    \item Beweis:  
      \begin{align*}
        x_{n+1} &= x_n + 2 &&\text{ nach Definition}\\
        &= 2n  + 2    && \text{ nach Induktionsvoraussetzung} \\
        &= 2(n+1) && \text{ fertig}\\
      \end{align*}
  \end{itemize}
\end{frame}

\begin{frame}{Klausuraufgabe: SS 2014 A5}
  Gegeben sei eine natürliche Zahl $a\in \mathbb{N}_+$. Die Abbildung $S:\mathbb{N}_0\rightarrow \mathbb{Z}$ sei induktiv definiert durch 
  \begin{align*}
    S(0)&=1,\\
    \forall k\in\mathbb{N}_0:S(k+1)&=a^{k+1}+S(k).
  \end{align*}
  Beweisen Sie durch vollständige Induktion, dass gilt:
  \begin{align*}
    \forall k\in\mathbb{N}_0:(a-1)S(k)=a^{k+1}-1.
  \end{align*}
\end{frame}

\begin{frame}[t]{Klausuraufgabe: SS 2014 A5}
  \begin{align*}
    S(0)&=1\\
    \forall k\in\mathbb{N}_0:S(k+1)&=a^{k+1}+S(k)\\
    \text{zu zeigen }\forall k\in\mathbb{N}_0:(a-1)S(k)&=a^{k+1}-1
  \end{align*}
  \pause
  \textbf{Induktionsanfang}\\
  $k=0$: Dann ist $(a-1)S(0)=a-1=a^{0+1}-1$
\end{frame}

\begin{frame}[t]{Klausuraufgabe: SS 2014 A5}
  \begin{align*}
    S(0)&=1\\
    \forall k\in\mathbb{N}_0:S(k+1)&=a^{k+1}+S(k)\\
    \text{zu zeigen }\forall k\in\mathbb{N}_0:(a-1)S(k)&=a^{k+1}-1
  \end{align*}
  \textbf{Induktionsvoraussetzung}\\
  für ein beliebiges aber festes $k$ gelte: $(a-1)S(k)=a^{k+1}-1$
\end{frame}

\begin{frame}[t]{Klausuraufgabe: SS 2014 A5}
  \begin{align*}
    S(0)&=1\\
    \forall k\in\mathbb{N}_0:S(k+1)&=a^{k+1}+S(k)\\
    \text{zu zeigen }\forall k\in\mathbb{N}_0:(a-1)S(k)&=a^{k+1}-1
  \end{align*}
  \textbf{Induktionsschluss $k\rightarrow k+1$}\\
  zu zeigen: $(a-1)S(k+1)=a^{(k+1)+1}-1$
  \pause
  \begin{align*}
    \action<+->{(a-1)S(k+1)&=(a-1)(a^{k+1}+S(k)) &&\text{ nach Definition}\\}
    \action<+->{&= (a-1)a^{k+1}+(a-1)S(k)\\}
    \action<+->{&= (a-1)a^{k+1}+a^{k+1}-1 && \text{ nach I.V.}\\}
    \action<+->{&= a^{(k+1)+1}-1}
  \end{align*}
\end{frame}

\begin{frame}{Klausuraufgabe: WS 2012/13 A3}
  Gegeben sei folgende Funktion $f:\{a,b\}^*\rightarrow\{a,b\}^*$:
  \begin{align*}
    f(\epsilon)&=\epsilon\\
    \forall w\in\{a,b\}^*: f(aw)&=bf(w)\\
    \forall w\in\{a,b\}^*: f(bw)&=af(w)
  \end{align*}
  Beweisen Sie per Induktion, dass gilt:
  \begin{align*}
    \forall w_1,w_2\in\{a,b\}^*: f(w_1w_2)&=f(w_1)f(w_2)
  \end{align*}
\end{frame}

\begin{frame}[t]{Klausuraufgabe: WS 2012/13 A3}
  \begin{align*}
    f(\epsilon)&=\epsilon\\
    \forall w\in A^*: f(aw)&=bf(w)\\
    \forall w\in A^*: f(bw)&=af(w)\\
    \text{zu zeigen }\forall w=w_1w_2\in A^n: f(w_1w_2)&=f(w_1)f(w_2)
  \end{align*}
  \pause
  \textbf{Induktionsanfang}\\
  $n=0$: $\{a,b\}^0=\{\epsilon\}$\\
  \pause
  $w_1=w_2=\epsilon$\\
  \pause
  $f(\epsilon\epsilon)=f(\epsilon)=\epsilon=\epsilon\epsilon=f(\epsilon)f(\epsilon)$
\end{frame}

\begin{frame}[t]{Klausuraufgabe: WS 2012/13 A3}
  \begin{align*}
    f(\epsilon)&=\epsilon\\
    \forall w\in A^*: f(aw)&=bf(w)\\
    \forall w\in A^*: f(bw)&=af(w)\\
    \text{zu zeigen }\forall w=w_1w_2\in A^n: f(w_1w_2)&=f(w_1)f(w_2)
  \end{align*}
  \textbf{Induktionsvoraussetzung}\\
  Für alle Wörter $w'$ mit beliebiger, aber fester Länge $n\in\mathbb{N}_0$ gelte:\\
  $\forall w'\in A^*$ mit $w'=w_1w_2: f(w_1w_2)=f(w_1)f(w_2)$
\end{frame}

\begin{frame}[t]{Klausuraufgabe: WS 2012/13 A3}
  \begin{align*}
    f(\epsilon)&=\epsilon\\
    \forall w\in A^*: f(aw)&=bf(w)\\
    \forall w\in A^*: f(bw)&=af(w)\\
    \text{zu zeigen }\forall w=w_1w_2\in A^n: f(w_1w_2)&=f(w_1)f(w_2)
  \end{align*}
  \textbf{Induktionsschritt} beliebiges $w\in A^{n+1}$\\
  \pause
  \begin{itemize}
    \item $w=aw'$:
    \pause
    $f(w)=f(aw')=bf(w')=bf(w_1w_2)=bf(w_1)f(w_2)=f(aw_1)f(w_2)$
    \pause
    \item $w=bw'$: $f(w)=f(bw')=af(w')=af(w_1w_2)=af(w_1)f(w_2)=f(bw_1)f(w_2)$
  \end{itemize}
\end{frame}
